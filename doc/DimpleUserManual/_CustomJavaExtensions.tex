\section{Creating Custom Dimple Extensions \ifmatlab in Java\fi}
\label{sec:userJava}

\subsection{Creating a Custom Factor Function}

\ifmatlab
There are some cases in which it is desirable to add a custom factor function written in Java rather than MATLAB.  Specific cases where this is desirable are:
%
\begin{itemize}
\item A factor function is needed to support continuous variables that is not available as a Dimple built-in factor.
\item A MATLAB factor function runs too slowly when creating a factor table, where a Java implementation may run more quickly.
\end{itemize}
\fi

\ifjava
When a factor function function is needed to support continuous variables that is not available as a Dimple built-in factor, then it is necessary to create a custom factor function.
\fi

To create a custom factor function, you must create a Java class that extends the Dimple \texttt{FactorFunction} class.  When extending the \texttt{FactorFunction} class, the following method must be overwritten:
%
\begin{itemize}
\item \texttt{evalEnergy}: Evaluates a set of input values and returns an energy value (negative log of a weight value).
\end{itemize}


The user may extend other methods, as appropriate:

\begin{itemize}
%
\item Constructor: If a constructor is specified (for example, to pass constructor arguments), it must call the constructor of the super class.
%
\item \texttt{isDirected}: Indicates whether the factor function is directed.  If directed, then there are a set of directed outputs for which the marginal distribution for all possible input values is a constant.  If not overridden, this is assumed false.
%
\item \texttt{getDirectedToIndices}: If a factor function is directed, indicates which edge indices are the directed outputs (numbering from zero), returning an array of integers.  There are two forms of this method, which may be used depending on whether the set of directed outputs depends on the number of edges in the factor that uses this factor function (many factor functions support a variable number of edges).  If \texttt{isDirected} is overridden and can return \texttt{true}, then this method must also be overridden.
%
\item \texttt{isDeterministicDirected}: Indicates whether a factor function is both directed and deterministic.  If deterministic and directed, then it is in the form of a deterministic function such that for all possible settings of the input values there is exactly one output value the results in a non-zero weight (or, equivalently, a non-infinite energy)\footnote{The indication that a factor function is deterministic directed is used by the Gibbs solver, and is necessary for such factor functions to work when using the Gibbs solver.}.  If not overridden, this is assumed false.
%
\item \texttt{evalDeterministic}: If a factor function is directed and deterministic, this method evaluates the values considered the inputs of the deterministic function and returns the resulting values for the corresponding outputs.  Note that these are not the weight or energy values, but the actual values of the associated variables that are considered outputs of the deterministic function.  If \texttt{isDeterministicDirected} is overridden and can return \texttt{true}, then this method must also be overridden.
%
\item \texttt{eval}: Evaluates a set of input values and returns a weight instead of an energy value.  Overriding this method would only be useful if implementing this method can be done significantly more computationally efficiently than the default implementation, which calls evalEnergy and then computes $\exp(-energy)$.
%
\end{itemize}

The following is a very simple example of a custom factor function:

\begin{lstlisting}
import com.analog.lyric.dimple.factorfunctions.core.FactorFunction;

/*
 * This factor enforces equality between all variables and weights
 * elements of the domain proportional to their value
 */
public class BigEquals extends FactorFunction
{	
    @Override
    public final double evalEnergy(Value[] input)
    {
        if (input.length == 0)
            return 0;
	    	
        Value firstVal = input[0];
	    	
        for (int i = 1; i < input.length; i++)
            if (!input[i].valueEquals(firstVal))
                return Double.POSITIVE_INFINITY;
	    	
        return 0;
    }
}
\end{lstlisting}




\subsection{Creating a Custom Proposal Kernel}
\label{sec:CreatingACustomProposalKernel}

In some cases, it may be useful to add a custom proposal kernel when using the Gibbs solver with a Metropolis-Hastings sampler.  In particular, since the \emph{block} Metropolis-Hastings sampler does not have a default proposal kernel, it is necessary to add a custom proposal kernel in this case.

To create a custom proposal kernel, you must create a Java class that implements either the Dimple \texttt{IProposalKernel} interface in the case of a single-variable proposal kernel, or the \texttt{IBlockProposalKernel} interface in the case of a block proposal kernel.

These interfaces define the following methods that must be implemented:
%
\begin{itemize}
%
\item \texttt{next} This method takes the current value(s) of the associated variable(s) along with the corresponding variable domain(s), and returns a proposal.  The proposal object returned (either a \texttt{Proposal} object for a single-variable proposal or a \texttt{BlockProposal}: object for a block proposal) includes both the proposed value(s) as well as the forward and reverse proposal probabilities (the negative log of these probabilities).
%
\item \texttt{setParameters}: Allows the class to include user-specified parameters that can be set to modify the behavior of the kernel.  This method must be implemented but may be empty if no parameters are needed.
%
\item \texttt{getParameters}: Returns the value of any user-specified parameters that have been specified.  This method must be implemented but may return null if no parameters are needed.
%
\end{itemize}


\ifmatlab

\subsection{Compiling Dimple Extensions in Java}

The new class must be compiled to class files, or optionally create a jar file. If using Eclipse, users can simply create a new project, create the new class, and the .class files will be created automatically.


\subsection{Adding Java Binary to MATLAB Path}

In MATLAB, you must use the javaaddpath call to add the Java class or jar files to the javaclasspath.  For example:

\begin{lstlisting}
javaaddpath('<path to my project>/MyFactorFunctions/bin');
\end{lstlisting}

or

\begin{lstlisting}
javaaddpath('<path to the jar>/myjar.jar');
\end{lstlisting}

\fi