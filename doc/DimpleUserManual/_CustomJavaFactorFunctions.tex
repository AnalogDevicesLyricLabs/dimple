\section{Creating Custom Java Factor Functions}
\label{sec:userJava}

There are some cases in which it is desirable to add a factor function that is defined in Java rather than MATLAB.  Specific cases where this is desirable are:

\begin{itemize}
\item A factor function is needed to support real variables that is not available as a Dimple built-in factor.
\item A MATLAB factor function runs too slowly when creating a factor table, where a Java implementation may run more quickly.
\end{itemize}

The following sections provides the steps users must follow to add a Java FactorFunction to their FactorGraph.

\subparaNoToc{Create a Class That Inherits from Java FactorFunction}
\label{sec:createJavaFactorFunction}

Users must extend the FactorFunction class. They have to provide two methods:

\begin{itemize}
\item A constructor that call's the parent constructor.
\item An eval function that returns a weight for each possible set of inputs \emph{or} an evalEnergy function that returns the energy (negative log of the weight) for each possible set of inputs\footnote{The evalEnergy function is preferred.}.
\end{itemize}

\begin{lstlisting}
import com.analog.lyric.dimple.factorfunctions.core.FactorFunction;

/*
 * This factor enforces equality between all variables and weights
 * elements of the domain proportional to their value
 */
public class BigEquals extends FactorFunction
{	
    public BigEquals() 
    {
	super("BigEquals");
    }
  
    @Override
    public double eval(Object... input) throws Exception 
    {
	if (input.length == 0)
	    return 0;
	else
	{
	    double first = (Double) input[0];
	    
	    for (int i = 1; i < input.length; i++)
	    {
	        if ((Double)input[i] != first)
		    return 0;
	    }
	    return first;   
	}
    }
}
\end{lstlisting}

\subparaNoToc{Compiling}

The new class must be compiled to class files. Users can optionally create a jar file. If using Eclipse, users can simply create a new project, create the new class, and the .class files will be created automatically.

\subparaNoToc{Adding Binary to MATLAB Path}

In MATLAB, the user must use the javaaddpath call to add the java files to the javaclasspath.

\begin{lstlisting}
javaaddpath('<path to my project>/MyFactorFunctions/bin');
\end{lstlisting}

or

\begin{lstlisting}
javaaddpath('<path to the jar>/myjar.jar');
\end{lstlisting}

