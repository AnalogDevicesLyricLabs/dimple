\subsection{Factors and Related Classes}

\subsubsection{Factor}

The Factor class can represent either a single factor or a multidimensional array of factors.  The Factor class is never created directly, but is the result of using the addFactor or addFactorVectorized (or related) methods on a FactorGraph.

\para{Properties}

\subpara{Name}

Read-write.  When read, retrieves the current name of the factor or array of factors.  When set, modifies the name of the factor to the corresponding value.  The value set must be a string.

\begin{lstlisting}
factor.Name = 'string';
\end{lstlisting}

When setting the Name, only one factor in an array may be set at a time.  To set the names of an entire array of factors to distinct values, the setNames method may be used (see section~\ref{sec:Factor.setNames}).


\subpara{DirectedTo}
\label{sec:Factor.DirectedTo}

Read-write.  The DirectedTo property indicates a set of variables to which the factor is directed.  The value may be either a single variable or a cell array of variables.  The DirectedTo property is used by some solvers, and in some cases is required for proper operation of certain features.  Such cases are identified elsewhere in this manual.

For example, if a factor F corresponds to a function $F(a, b, c, d)$, where $a$, $b$, $c$, and $d$ are variables, then the factor is directed toward $c$ and $d$ if $\sum_{c, d} F(a, b, c, d)$ is constant for all values of $a$ and $b$.  In this case, we may set:

\begin{lstlisting}
F.DirectedTo = {c, d};
\end{lstlisting}

In some cases, the set of DirectedTo variables can be automatically determined when a factor is created.  In this case it need not be set manually by the user.  This includes many built-in factors supported by Dimple.  If this property is set by the user, then in the case of factors connected only to discrete variables, Dimple will check that the factor is in fact directed in the specified direction.

If the set of variables the factor are directed toward are part of a variable array, then these may be specified together in a single cell array.  For example, if varArray is an array of variables, and a factor F is directed toward all of the variables in varArray, then we can set:

\begin{lstlisting}
F.DirectedTo = varArray;
\end{lstlisting}

In the case of a vector of factors, we can identify the variables to which each factor is directed in a vectorized way.  For example:

\begin{lstlisting}
s = Discrete(domain,N);
fg.addFactorVectorized(factorFunction, s(1:(end-1)), s(2:end)).DirectedTo = s(2:end);
\end{lstlisting}

This example also shows that the DirectedTo property can be set directly on the result of the factor creation without assigning the factor to a named variable.

As a more complicated vectorized example, the following creates 12 factors, each of which contains 10 variables (5 from a and 5 from b).  The first 2 of the 5 from a and the first from b are what the factor is directed to.

\begin{lstlisting}
a = Bit(3,4,5);
b = Bit(3,4,5);
fg.addFactorVectorized(factorFunction, {a, [1 2]}, {b, [1 2]}).DirectedTo = {a(:,:,1:2), b(:,:,1)};
\end{lstlisting}

\subpara{Belief}
\label{sec:Factor.Belief}

Read-only.  The belief of a factor is the joint belief over all joint states of the variables connected to that factor.  There are two properties that represent the belief in different ways: Belief and FullBelief.  Reading the Belief property after the solver has been run\footnote{In the current version of Dimple, the Belief property is only supported for factors connected exclusively to discrete variables, and is supported only by the SumProduct solver.  These restrictions may be removed in a future version.} returns the belief in a compact one-dimensional vector that includes only values that correspond to non-zero entries in the factor table.  This form is used because in some situation, the full representation over all possible variable values (as returned by the FullBelief property) would result in a data structure too large to be practical.

\begin{lstlisting}
fb = myFactor.Belief;
\end{lstlisting}

The result is a vector of belief values, where the order of the vector corresponds to the order of the factor table entries.  The order of factor table entries can be determined from the factor using:

\begin{lstlisting}
ind = f.FactorTable.Indices
\end{lstlisting}

This returns a two-dimensional array, where each row corresponds to one entry in the factor table, and where each column-entry in a row indicates the index into the domain of the corresponding variable (where the order of the variable is the order used when the factor was created).


\subpara{FullBelief}
\label{sec:Factor.FullBelief}

Read-only.  Reading the FullBelief property after the solver has been run\footnote{In the current version of Dimple, the Belief property is only supported for factors connected exclusively to discrete variables, and is supported only by the SumProduct solver.  These restrictions may be removed in a future version.} returns the belief in a multi-dimensional array, where each dimension of the multi-dimensional array ranges over the domain of the corresponding variable (the order of the dimensions corresponds to the variable order used when the factor was created).

\begin{lstlisting}
fb = myFactor.FullBelief;
\end{lstlisting}


\subpara{Score}
\label{sec:Factor.Score}

Read-only.  When read, computes and returns the score (energy) of the factor given a specified value for each of the neighboring variables to this factor.  The score represents the energy of the factor given the specified variable configuration.  The score value is relative, and may be arbitrarily normalized by an additive constant.

The value of each variable used when computing the Score is the Guess value for that variable (see section~\ref{sec:Variable.Guess}).  If no Guess had yet been specified for a given variable, the value with the most likely belief (which corresponds to the Value property of the variable) is used\footnote{For some solvers, beliefs are not supported for all variable types; in such cases there is no default value, so a Guess must be specified.}.


\subpara{Ports}

Read-only.  Retrieves a cell array containing a list of Ports connecting the factor to its neighboring variables.


\para{Methods}

\subpara{setNames}
\label{sec:Factor.setNames}

For an array of factors, the setNames method sets the name of each factor in the array to a distinct value derived from the supplied string argument.  When called with:

\begin{lstlisting}
factorArray.setName('baseName');
\end{lstlisting}

the resulting factor names are of the form: \texttt{baseName\textunderscore vv0}, \texttt{baseName\textunderscore vv1}, \texttt{baseName\textunderscore vv2}, etc., where each factor's name is the concatenation of the base name with the suffix \texttt{\textunderscore vv} followed by a unique number for each factor in the array.


\subpara{invokeSolverSpecificMethod}

\begin{lstlisting}
factorArray.invokeSolverSpecificMethod('methodName', arguments);
\end{lstlisting}

For an array of factors, the invokeSolverSpecificMethod calls the specified solver-specific method on each of the factors in the array.  Here, 'methodName' is a text string with the name of the solver-specific method, and arguments is an optional comma-separated list of arguments to that method.  This method does not result in any return values.  For solver specific methods that return results, use the invokeSolverSpecificMethodWithReturnValue method instead.


\subpara{invokeSolverSpecificMethodWithReturnValue}

\begin{lstlisting}
returnArray = factorArray.invokeSolverSpecificMethodWithReturnValue('methodName', arguments);
\end{lstlisting}

For an array of factors, the invokeSolverSpecificMethodWithReturnValue calls the specified solver-specific method on each of the factors in the array, returning a return value for each variable.  Here, 'methodName' is a text string with the name of the solver-specific method, and arguments is an optional comma-separated list of arguments to that method.  The returnArray is a cell-array of the return values of the method, with dimensions equal to the dimensions of the factor array.




\subsubsection{FactorFunction}
\label{sec:FactorFunction}

The FactorFunction class is used specify a Dimple built-in factor function in a way that can be reused for creating multiple factors, and that allows specification of constructor arguments.

\para{Constructor}

\begin{lstlisting}
FactorFunction(factorFunctionName, [constructorArguments])
\end{lstlisting}

\begin{itemize}
\item factorFunctionName is a string that indicates the name of the built-in factor function.
\item constructorArguments is a variable-length comma-separated list of constructor arguments, whose interpretation is specific to the particular built-in factor function.  If no arguments are needed, this list would be empty.
\end{itemize}

There are no available properties or methods in this class.



\subsubsection{FactorTable}
\label{sec:FactorTable}

The FactorTable class is used to explicitly specify a factor table in lieu of Dimple creating one automatically from a factor function.  This is sometimes useful in cases where the factor table is highly structured, but automatic creation would be time consuming due to a large number of possible states of the connected variables.

\para{Constructor}

\begin{lstlisting}
FactorTable([ ( indexList, weightList ) | weightMatrix ], domains)
\end{lstlisting}

A FactorTable is constructed by specifying the table values in one of two forms, or by creating an all-zeros FactorTable to be filled in later using the set method.  The first form, specifying an indexList and weightList, is useful for sparse factor tables in which many entires are zero and need not be included in the table.  The second form, specifying a weightMatrix, is useful for dense factor tables in which most or all of the entries are non-zero.

\begin{itemize}
\item indexList is an array where each row represents a set of zero-based indices into the list of domain elements for each successive domain in the given set of domains.
\item weightList is a one-dimensional array of real-valued entries in the factor table, where each entry corresponds to the indices given by the corresponding row of the indexList.
\item weightArray is an N dimensional array of real-valued entries in the factor table.  The number of dimensions, N, must correspond to the number of discrete domain elements given in the subsequent arguments, and the number of elements in each dimension must equal the number of elements in the corresponding domain element.
\item domains is a comma-separated list of one or more discrete variable domains.  Each domain either in the form of a DiscreteDomain object or a cell-array of the elements of the domain.  From an existing variable, the domain can be obtained using the Domain property of that variable.
\end{itemize}

An example using the first method is:
\begin{lstlisting}
a = Bit;
b = Bit;
c = Bit;
ft = FactorTable([0 0 0; 0 1 1; 1 0 1; 1 1 0], [1 .9 .5 .3], a.Domain, b.Domain, c.Domain);
\end{lstlisting}

An example using the second method is:
\begin{lstlisting}
a = Discrete({'red','green','blue'});
b = Bit;
ft = FactorTable([.5 .4 .3; .1 0 .4], a.Domain, b.Domain);
\end{lstlisting}


\para{Properties}

\subpara{Indices}

Read-write.  When read, returns an array of indices of the factor table corresponding to entries in the factor table.  Each row represents a set of zero-based indices into the list of domain elements for each successive domain in the given set of domains.

When written, replaces the previous array of indices with a new array.  When writing using this property, the number of rows in the table must not change since this must equal the number of entires in the Weights.  To change both Indices and Weights simultaneously, use the change method.

\subpara{Weights}

Read-write.  When read, returns a one-dimensional array of real-valued entries in the factor table, where each entry corresponds to the indices given by the corresponding row of the indexList.

When written, replaces the previous array of weights.  When writing using this property, the number of entries must not change since this must equal the number of rows in the Indices.  To change both Indices and Weights simultaneously, use the change method.

\subpara{Domains}

Read-only.  Returns a cell-array of DiscreteDomain objects, each of which represents the corresponding domain specified in the constructor, in the same order as specified in the constructor.


\para{Methods}

\subpara{set}
\label{sec:FactorTable.set}

This method allows setting individual entries in the factor table.  For each entry to be set, the domain values (not indices) are specified followed by the weight value.  To set a single entry, the domain values are specified in a comma-separated list, followed by the weight value.  To set multiple entries in a single call, a cell array of such comma-separated lists are specified.

An example of setting a single entry:
\begin{lstlisting}
ft.set('red', 1, 0.45);
\end{lstlisting}

An example of setting a single entry:
\begin{lstlisting}
ft.set({'red', 1, 0.45}, {'blue', 0, 0.75});
\end{lstlisting}


\subpara{get}

This method retrieves the weight associated with a particular entry in the factor table.  The entry is specified by a comma-separated list of domain values (not indices).  For example:

\begin{lstlisting}
w = ft.get('red', 1);
\end{lstlisting}


\subpara{change}

This method allows simultaneously replacing both the array of Indices and Weights in the factor table.

\begin{lstlisting}
ft.change(indexList, weightList);
\end{lstlisting}

The arguments indexList and weightList are exactly as described in the FactorTable constructor.




%\subsubsection{Factor Belief}
%
%
%Users can retrieve beliefs from Factors. As demonstrated in this unit test:
%
%\begin{lstlisting}
%function output = funkyFactor(x,y,z)
%   if x == y && y == z
%       if x == 0
%           output = 1;
%       else
%           output = 2;
%       end
%   else
%       output = 0;
%   end
%end
%    %Create a Factor Graph
%       b = Bit(3,1);
%       fg = FactorGraph();
%       f = fg.addFactor(@funkyFactor,b(1),b(2),b(3));
%       
%       %Set inputs
%       input = [.8 .8 .6];
%       b.Input = input;
%       %We have to solver right now
%       fg.Solver.setNumIterations(1);
%       fg.solve();
%       %funkyFactor only allows these two combos.
%       expectedDomain = {0,0,0;1,1,1};
%       assertEqual(expectedDomain,f.Domain);
%       %Bit stores values as 1,0.
%       expectedIndices = int32([2,2,2;1,1,1]);
%       assertEqual(expectedIndices,f.Indices);
%       %Let's calculate the belief by hand and compare to Dimple.
%       val0 = prod(1-input)*funkyFactor(0,0,0);
%       val1 = prod(input)*funkyFactor(1,1,1);
%       total = val0 + val1;
%       val0 = val0 / total;
%       val1 = val1 / total;
%       expectedBelief = [val0; val1];
%       assertElementsAlmostEqual(expectedBelief,f.Belief);
%       %Now we create a full belief and compare to our expected.
%      expectedFullBelief = zeros(2,2,2);
%       expectedFullBelief(1,1,1) = val1;
%       expectedFullBelief(2,2,2) = val0;
%       assertElementsAlmostEqual(expectedFullBelief,f.FullBelief);
%\end{lstlisting}
%
%%%%%%%%%%%%%%%%%%%%%
