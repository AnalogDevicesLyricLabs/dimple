\subsection{Streaming Data}

\subsubsection{Variable Stream Common Properties and Methods}

Dimple supports several types of variable streams:

\begin{itemize}
\item DiscreteStream
\item BitStream
\item RealStream
\item RealJoinStream
\end{itemize}

Each of these share common properties and method listed in the following sections.

\para{Properties}

\subpara{DataSource}

Read-write.  When written, connects the variable stream to a data source (see section~\ref{sec:DataSource}).  The data sink must be of a type appropriate for the particular variable stream type.

\subpara{DataSink}

Read-write.  When written, connects the variable stream to a data sink (see section~\ref{sec:DataSink}).  The data sink must be of a type appropriate for the particular variable stream type.

\subpara{Dimensions}

Read-only.  Indicates the dimensions of the variable stream.  The dimensions correspond to the size of the variable array at each position in the stream.

\subpara{Size}

Read-only.  Indicates the number of elements in the variable stream that are actually instantiated.  Each element corresponds to one copy of the variable or variable array at a specific point in the stream.  Dimple instantiates the minimum number of contiguous elements to cover the slices of the stream that are actually used in factors (see section~\ref{sec:VariableStream.getSlice}), plus the number of additional elements to cover the indicated BufferSize (see section~\ref{sec:FactorGraphStream.BufferSize}).

\subpara{Variables}

Read-only.  Returns a variable array containing all of the currently instantiated variables in the stream.


\para{Methods}

\subpara{getSlice}
\label{sec:VariableStream.getSlice}

\begin{lstlisting}
varStream.getSlice(startIndex);
\end{lstlisting}

The getSlice method is used to extract a \emph{slice} of the stream, which means a version of the stream that may be offset from the original stream itself.  This is generally used for specifying streams to connect to a factor when calling addFactor.

Takes a single numeric argument, startIndex, which indicates the starting position in the stream.  The resulting stream slice is essentially a reference to the stream offset by startIndex-1.  For example, a startIndex of 2 returns a slice offset by 1, such that the first location in the slice corresponds to the second location in the original stream.  A startIndex of 1 returns a slice identical to the original stream.


\subsubsection{DiscreteStream}

\para{Constructor}
\label{DiscreteStream.Constructor}

The DiscreteStream constructor is used to create a stream of Discrete variables or arrays of Discrete variables.

\begin{lstlisting}
DiscreteStream(domain, [dimensions]);
\end{lstlisting}

\begin{itemize}
\item domain is a required argument indicating the domain of the variable.  The domain may either be a numeric array of domain elements, a cell array of domain elements, or a DiscreteDomain object (see section~\ref{Discrete.Domain}).
\item dimensions is an optional variable-length comma-separated list of matrix dimensions (an empty list indicates a single Discrete variable).
\end{itemize}

\subsubsection{BitStream}

\para{Constructor}

The BitStream constructor is used to create a stream of Bit variables or arrays of Bit variables.

\begin{lstlisting}
BitStream([dimensions]);
\end{lstlisting}

\begin{itemize}
\item dimensions is an optional variable-length comma-separated list of matrix dimensions (an empty list indicates a single Bit variable stream).
\end{itemize}

\subsubsection{RealStream}

\para{Constructor}

The RealStream constructor is used to create a stream of Real variables or arrays of Real variables.

\begin{lstlisting}
RealStream([dimensions]);
\end{lstlisting}

\begin{itemize}
\item dimensions is an optional variable-length comma-separated list of matrix dimensions (an empty list indicates a single Real variable stream).
\end{itemize}

\subsubsection{RealJoinStream}

\para{Constructor}

The RealJoinStream constructor is used to create a stream of RealJoint variables or arrays of RealJoint variables.

\begin{lstlisting}
RealJoinStream(numJointVariables, dimensions);
\end{lstlisting}

\begin{itemize}
\item dimensions is an optional variable-length comma-separated list of matrix dimensions (an empty list indicates a single RealJoint variable stream).
\end{itemize}


\subsubsection{FactorGraphStream}

A FactorGraphStream is constructed automatically and returned as the result of adding a factor to a graph using the addFactor method where one or more of the arguments are variable streams.

\para{Properties}

\subpara{BufferSize}
\label{sec:FactorGraphStream.BufferSize}

Read-write. When written, modifies the number of instantiated elements in the FactorGraphStream to include the specified number of copies of the corresponding factor and connected variables.  By default, the BufferSize is 1.  When running the solver on one step of the overall factor graph, the solver uses the entire buffer.  Making the buffer size larger means using more of the history in performing inference for each step.  The results of inference run on previous steps that is beyond the size of the buffer is essentially frozen, and is no longer updated on subsequent steps of the solver.


\subsubsection{Data Source Common Properties and Methods}
\label{sec:DataSource}

Dimple supports several types of streaming data sources.  A data source is a source of Input values to the variables within a variable stream\footnote{In the current version of Dimple, data sources are limited to providing Inputs to variables.  A future version of Dimple may expand this capability to allow sourcing FixedValues or other types of input data.}.  When performing inference, as each step that the graph is advanced, the next source value is read from the data source, and the earlier values are shifted back to earlier time-steps in the graph.

Each source type corresponds to a particular format of input data.  Each type is appropriate only to a specific type of variable stream and solver.  The following table summarizes these requirements.

\begin{longtable} {l | l | l}
Data Source & Variable Stream Type & Supported Solvers \\
\hline
\endhead
DoubleArrayDataSource & DiscreteStream, BitStream & all \\
 & RealStream & Gaussian \\
MultivariateDataSource & RealJointStream & Gaussian \\
FactorFunctionDataSource & RealStream & Gibbs, ParticleBP \\
\end{longtable} 

Each of these share common properties and method listed in the following sections.

\para{Properties}

\subpara{Dimensions}

Read-only.  Indicates the dimensions of the data source.  The dimensions correspond to the size of the variable array at each position in the stream that the data source will feed.


\subsubsection{DoubleArrayDataSource}

\para{Constructor}

\begin{lstlisting}
DoubleArrayDataSource([dimensions], [initialData]);
\end{lstlisting}

\begin{itemize}
\item dimensions is an optional row vector indicating the matrix dimensions of the data source (an empty argument indicates a single data source to connect to a single variable stream).
\item initialData is an optional array of data the stream will contain (more data can be added later).  This is a multidimensional array where the first dimensions correspond to the dimension of the variable stream this will feed, the next dimensions corresponds to the length of the Input vector for each variable (the domain size for discrete variable streams, and 2 for Gaussian variable streams), and the final dimension is the number of time-steps of data to provide.  For single variable streams, the first dimensions are omitted.
\end{itemize}

\para{Methods}

\subpara{add}

\begin{lstlisting}
dataSource.add(data);
\end{lstlisting}

This method appends the data source with the specified data.  The data argument is a multidimensional array, where the first dimensions correspond to the dimension of the variable stream this will feed, the next dimensions corresponds to the length of the Input vector for each variable (the domain size for discrete variable streams, and 2 for Gaussian variable streams), and the final dimension is the number of time-steps of data to provide.  For single variable streams, the first dimensions are omitted.

\subsubsection{MultivariateDataSource}

\para{Constructor}

\begin{lstlisting}
MultivariateDataSource([dimensions]);
\end{lstlisting}

\begin{itemize}
\item dimensions is an optional row vector indicating the matrix dimensions of the data source (an empty argument indicates a single data source to connect to a single variable stream).
\end{itemize}

\para{Methods}

\subpara{add}

\begin{lstlisting}
dataSource.add(means, covariances);
\end{lstlisting}

This method appends the data source with the a single time-step of data (multiple time-steps must be added using successive calls to this method).  Means should have dimensions [VariableDimensions NumberJointVariables] and Covariances should be of the form [VariableDimensions NumberJointVariables NumberJointVariables].

\subsubsection{FactorFunctionDataSource}

\para{Constructor}

\begin{lstlisting}
FactorFunctionDataSource([dimensions]);
\end{lstlisting}

\begin{itemize}
\item dimensions is an optional row vector indicating the matrix dimensions of the data source (an empty argument indicates a single data source to connect to a single variable stream).
\end{itemize}

\para{Methods}

\subpara{add}

\begin{lstlisting}
dataSource.add(data);
\end{lstlisting}

This method appends the data source with the a single time-step of data (multiple time-steps must be added using successive calls to this method).The data argument is a multidimensional cell array, with dimension equal to the corresponding dimensions of the variable stream this will feed.  Each element is a FactorFunctions (see section~\ref{sec:FactorFunction}), which is to represent the Input of the corresponding variable.


\subsubsection{Data Sink Common Properties and Methods}
\label{sec:DataSink}

Dimple supports several types of streaming data sinks:

Dimple supports several types of streaming data sinks.  A data sink is a data structure used to store successive results of inference from the variables with a variable stream.  Specifically, it stores the Belief values of these variables\footnote{In the current version of Dimple, data sinks are limited to the Beliefs to variables.  A future version of Dimple may expand this capability to allow sinking other types of result data.}.  When performing inference, as each step that the graph is advanced, the Belief value for the earliest element of the variable stream is stored in the data sink.

Each sink type corresponds to a particular format of output data.  Each type is appropriate only to a specific type of variable stream and solver.  The following table summarizes these requirements.

\begin{longtable} {l | l | l}
Data Sink & Variable Stream Type & Supported Solvers \\
\hline
\endhead
DoubleArrayDataSink & DiscreteStream, BitStream & all \\
 & RealStream & Gaussian \\
MultivariateDataSink & RealJointStream & Gaussian \\
\end{longtable} 

Each of these share common properties and method listed in the following sections.

\para{Properties}

\subpara{Dimensions}

Read-only.  Indicates the dimensions of the data sink.  The dimensions correspond to the size of the variable array at each position in the stream that the data sink will be fed from.

\para{Methods}

\subpara{hasNext}

\begin{lstlisting}
hasNext = dataSink.hasNext();
\end{lstlisting}

Used in connection with the getNext method (described in the sections below), this method takes no arguments and returns a boolean indicating whether or not there are any more time steps in the dataSink that have not yet been extracted.



\subsubsection{DoubleArrayDataSink}
\para{Constructor}

\begin{lstlisting}
DoubleArrayDataSink([dimensions]);
\end{lstlisting}

\begin{itemize}
\item dimensions is an optional array indicating the matrix dimensions of the data sink (an empty argument indicates a single data sink to connect to a single variable stream).
\end{itemize}

\para{Properties}

\subpara{Array}

Read-only.  Extracts the entire contents of the data sink as an array.  The first dimensions of the array correspond to the Dimensions of the data sink, the next dimension corresponds to the size of the belief array for each variable, and the final dimension corresponds to the number of time steps that had been gathered.  For discrete variables, the dimension of the belief array corresponds to the domain sizes, while for Gaussian variables the dimension is 2, where the elements correspond to the mean and standard deviation, respectively.

\para{Methods}

\subpara{getNext}

\begin{lstlisting}
b = dataSink.getNext();
\end{lstlisting}

This method takes no arguments, and returns the set of belief values from the next time-step.  The returned value is a multidimensional array, where the first dimensions correspond to the Dimension of the data sink, and the next dimension corresponds to the size of the belief array for each variable.  For discrete variables, the dimension of the belief array corresponds to the domain sizes, while for Gaussian variables the dimension is 2, where the elements correspond to the mean and standard deviation, respectively.


\subsubsection{MultivariateDataSink}

\para{Constructor}

\begin{lstlisting}
MultivariateDataSink([dimensions]);
\end{lstlisting}

\begin{itemize}
\item dimensions is an optional array indicating the matrix dimensions of the data sink (an empty argument indicates a single data sink to connect to a single variable stream).
\end{itemize}

\para{Methods}
\subpara{getNext}

\begin{lstlisting}
b = dataSink.getNext();
\end{lstlisting}

This method takes no arguments, and returns the set of belief values from the next time-step.  The returned value is a cell array of MultivariateMsg objects (see section~\ref{sec:MultivariateMsg}), each of which contains and mean vector and covariance matrix.  The dimensions of this cell array correspond to the Dimension of the data sink.


