\subsection{Variables and Related Classes}

\subsubsection{Variable Types}

The following variable types are defined in Dimple.  Some variable types are supported by only a subset of solvers.  The following table lists the Dimple variable types and the solvers that support them.

\begin{longtable} {l | p{5cm}}
Variable Type & Supported Solvers \\
\hline
\endhead
Discrete & all \\
Bit & all \\
Real & Gibbs, Gaussian, ParticleBP \\
RealJoint & Gibbs, Gaussian \\ 
Complex & Gibbs, Gaussian \\
FiniteFieldVariable & SumProduct \\
\end{longtable} 

\subsubsection{Common Properties and Methods}

The following properties and methods are common to variables of all types.

\para{Properties}

\subpara{Name}

Read-write.  When read, retrieves the current name of the variable or array of variables.  When set, modifies the name of the variable to the corresponding value.  The value set must be a string.

\ifmatlab
\begin{lstlisting}
var.Name = 'string';
\end{lstlisting}

When setting the Name, only one variable in an array may be set at a time.  To set the names of an entire array of variables to distinct values, the setNames method may be used (see section~\ref{sec:Variable.setNames}).

\fi

\ifjava
\begin{lstlisting}
var.setName('string');
\end{lstlisting}

\fi

\subpara{Label}

Read-write. All variables and factors in a Factor Graph must have unique names. However, sometimes it is desirable to have variables or factors share similar strings when being plotted or printed. Users can set the Label property to set the name for display. If the Label is not set, the Name will be used for display. Once the label is set, the label will be used for display.

\ifmatlab
\begin{lstlisting}
var.Label = 'string';
\end{lstlisting}
\fi

\ifjava
\begin{lstlisting}
var.setLabel('string');
\end{lstlisting}
\fi


\subpara{Solver}

Read-only.  Returns the solver-object associated with the variable, to which solver-specific methods can be called.  See section~\ref{sec:SolversAPI}, which describes the solvers, including the solver-specific methods for each solver.

\subpara{Guess}
\label{sec:Variable.Guess}

Read-write.  Specifies a value from the variable to be used when computing the Score of the factor graph (or of the variable or neighboring factors).  The Guess must be a valid value from the domain of the variable.

If the Guess had not yet been set, its value defaults to the most likely belief (which corresponds to the Value property of the variable)\footnote{For some solvers, beliefs are not supported for all variable types; in such cases there is no default value, so a Guess must be specified.}.


\subpara{Score}
\label{sec:Variable.Score}

Read-only.  When read, computes and returns the score (energy) of the Input to this variable, which is treated as a single-edge factors, given a specified value for the variable.  The score value is relative, and may be arbitrarily normalized by an additive constant.

The value of the variable used when computing the Score is the Guess value for this variable (see section~\ref{sec:Variable.Guess}).  If no Guess had yet been specified, the value with the most likely belief (which corresponds to the Value property of the variable) is used\footnote{For some solvers, beliefs are not supported for all variable types; in such cases there is no default value, so a Guess must be specified.}.


\subpara{Internal Energy}
\label{sec:Variable.InternalEnergy}

Read-only.  (Only applies to the Sum Product Solver).  When read, returns:

\[
InternalEnergy(i) = \sum_{d \in D}B_i(d)*(-log(Input(d))) 
\]

Read-only.  When read returns:

Where D is variable i's domain, Input is the variable's input, and $B_i$ is the variable Belief.

\ifjava
\begin{lstlisting}
double ie = v.getInternalEnergy();
\end{lstlisting}
\fi

\subpara{Bethe Entropy}
\label{sec:Variable.BetheEntropy}

Read-only.  (Only applies to the Sum Product Solver).  When read, returns:

\[
BetheEntropy(i) = - \sum{d \in D}B_i(d)*log(B_i(d))
\]

Where D is variable i's domain and $B_i$ is the variable Belief.

\ifjava
\begin{lstlisting}
double be = v.getBetheEntropy();
\end{lstlisting}
\fi


\subpara{Ports}

Read-only.  Retrieves \ifmatlab a cell array \fi \ifjava an array \fi containing a list of Ports connecting the variable to its neighboring factors.

\ifmatlab
\para{Methods}
\fi

\ifmatlab
\subpara{setNames}
\label{sec:Variable.setNames}

For an array of variables, the setNames method sets the name of each variable in the array to a distinct value derived from the supplied string argument.  When called with:

\begin{lstlisting}
varArray.setName('baseName');
\end{lstlisting}

the resulting variable names are of the form: \texttt{baseName\textunderscore vv0}, \texttt{baseName\textunderscore vv1}, \texttt{baseName\textunderscore vv2}, etc., where each variable's name is the concatenation of the base name with the suffix \texttt{\textunderscore vv} followed by a unique number for each variable in the array.
\fi

\ifmatlab
\subpara{invokeSolverSpecificMethod}

\begin{lstlisting}
variableArray.invokeSolverSpecificMethod('methodName', arguments);
\end{lstlisting}

For an array of variables, the invokeSolverSpecificMethod calls the specified solver-specific method on each of the variables in the array.  Here, 'methodName' is a text string with the name of the solver-specific method, and arguments is an optional comma-separated list of arguments to that method.  This method does not result in any return values.  For solver specific methods that return results, use the invokeSolverSpecificMethodWithReturnValue method instead.

\fi

\ifmatlab
\subpara{invokeSolverSpecificMethodWithReturnValue}

\begin{lstlisting}
returnArray = variableArray.invokeSolverSpecificMethodWithReturnValue('methodName', arguments);
\end{lstlisting}

For an array of variables, the invokeSolverSpecificMethodWithReturnValue calls the specified solver-specific method on each of the variables in the array, returning a return value for each variable.  Here, 'methodName' is a text string with the name of the solver-specific method, and arguments is an optional comma-separated list of arguments to that method.  The returnArray is a cell-array of the return values of the method, with dimensions equal to the dimensions of the variable array.

\fi

\ifmatlab
\para{Operators}

\subpara{Operators for Implicit Factor Creation}

Dimple supports a set of overloaded MATLAB operators and functions that operate on variables to implicitly add factors to a factor graph.

The list of supported operators and functions is given in section~\ref{sec:overloaded}.  A description of how to use these operators and functions is given in section~\ref{sec:ImplicitFactorCreation}.

Using one of the defined operators or functions with one or more variables will result in creation of a new variable, which can be assigned to a new variable name, with the appropriate domain.  It will also result in the creation of a factor which is added to the most recently created factor graph.  This factor will connect to the input variables and the newly created result variable.  For example:

\begin{lstlisting}
c = a + b;
\end{lstlisting}

This results in the creation of a Sum factor with connected variables, c, a, and b (in that order).

These operators and functions can be compounded into a single line of code, resulting in creation of intermediate anonymous variables.  For example:

\begin{lstlisting}
z = (a + b) * c^d - sqrt(-e);
\end{lstlisting}

Like using the addFactor method, some of the inputs to these operators and functions may be constants.  Specifically, for binary operators, one of the inputs may be a constant instead of a variable.  For example:

\begin{lstlisting}
x = a^2;
y = (a + b + 2) * 3;
z = a * (2 + 1i);
\end{lstlisting}

These operators and functions can be applied to arrays of variables, with some limitations.  Specifically, if each of the input variables are vectors of the same dimension, then the result will be to create a vector of output variables of the same dimension, along with a vector of factors relating the inputs and outputs.

In some cases, to be consistent with MATLAB notation, there is a distinction made between the vectorized and non-vectorized operator.  Specifically, Dimple uses MATLAB's notation for pointwise product and power operators to indicate a vectorized operation.  For example, if variables a through e are vectors of variables of identical size, then the following would create a variable vector z, and a series of factors relating these variables.

\begin{lstlisting}
z = (a .* b) + c.^d - sqrt(-e);
\end{lstlisting}

For binary operators, one of the inputs may be a scalar variable or a scalar constant instead of a variable vector.  For a scalar variable, the result is that scalar variable connecting to each instance of the factors that are created.  For a constant, each instance of the factor uses the same constant for that input (vectors of distinct constants are not currently supported).

\subpara{repmat}

Dimple overloads the MATLAB repmat function when called with a Dimple variable or variable array as its first argument.  The result of this function is an array of variables in the requested dimensions.  Dimple does not actually make multiple copies of the variables, but instead creates a variable array that provides repeated references to the same variables in the existing array.  

For example:
\begin{lstlisting}
var = Bit(10, 1);
varRef = repmat(var, 1, 10);
\end{lstlisting}

In this case varRef is a 10x10 array of variables.  Each element of varRef, however, is not a distinct Dimple variable, but a reference to an element in var, where for each row of varRef, there are 10 repeated copies of the corresponding variable in var.  Use of repmat for variables can be useful when using addFactorVectorized (see sections~\ref{sec:vectorizedFactorCreation} and~\ref{sec:FactorGraph.addFactorVectorized}).

\fi

\subsubsection{Discrete}
\label{sec:Discrete}

A Discrete variable represents a variable that can take on a finite set of distinct states.  The Discrete class corresponds to either a single Discrete variable or a multidimensional array of Discrete variables.  All properties/methods can either be called for all elements in the collection or for individual elements of the collection.

\para{Constructor}

The Discrete constructor can be used to create an N-dimensional collection of Dimple Discrete variables.  The constructor is called with the following arguments (arguments in brackets are optional).

\ifmatlab
\begin{lstlisting}
Discrete(domain, [dimensions])
\end{lstlisting}
\fi

\ifjava
\begin{lstlisting}
Discrete(domain)
\end{lstlisting}
\fi

\ifmatlab
\begin{itemize}
\item domain is a required argument indicating the domain of the variable.  The domain may either be a numeric array of domain elements, a cell array of domain elements, or a DiscreteDomain object (see section~\ref{Discrete.Domain}).
\item dimensions is an optional variable-length comma-separated list of matrix dimensions (an empty list indicates a single Discrete variable).
\end{itemize}
\fi

\ifjava
\begin{itemize}
\item domain is a required argument indicating the domain of the variable.  The domain may either be an object array of domain elements, a comma separated list, or a DiscreteDomain object (see section~\ref{Discrete.Domain}).
\end{itemize}
\fi

For example:

\ifmatlab
\begin{lstlisting}
domain = [0 1 2];
w = Discrete(domain);
x = Discrete(domain, 4);
y = Discrete(domain, 2, 3);
z = Discrete(domain, 2, 3, 4);
\end{lstlisting}
\fi

\ifjava
\begin{lstlisting}
Object [] domain = new Object [] {0,1,2};
Discrete w = new Discrete(domain);
Discrete x = new Discrete(0,1,2);
DiscreteDomain dd = DiscreteDomain.create(0,1,2);
Discrete y = new Discrete(dd);
\end{lstlisting}
\fi

We examine each of these arguments in more detail in the following sections.

\subpara{Domain}
\label{Discrete.Domain}

Every Discrete random variable has a domain associated with it.  A domain is a set.  Elements of the set may be any object type.  For example, the following are Discrete variables with valid domains:

\ifmatlab
\begin{lstlisting}
a = Discrete({1, 2, 3});
b = Discrete({1+i, i, 2*i});
c = Discrete({[1 0; 0 1], [i 1, 2*i 1]});
d = Discrete({[1 0; 0 1], 2, i+1});
e = Discrete({1.2, 3, pi/2});
f = Discrete({'red', 'green', 'blue'});
\end{lstlisting}
\fi

\ifjava
\begin{lstlisting}
Discrete a = new Discrete(1, 2, 3);
Discrete b = new Discrete(new double [] {1,0,0,1},1); 
Discrete c = new Discrete("red","blue",2);
\end{lstlisting}
\fi

\ifmatlab
(a) creates a variable whose domain consists of three values: 1, 2, and 3.  (b) creates a variable whose domain consists of three complex numbers.  (c) creates a variable whose domain consists of two elements, each of which is a 2x2 complex matrix.  (d) creates a variable whose domain consists of three elements: a matrix, real scalar, and complex scalar.  (e) creates a variable whose domain consists of both floating-point and integer values.  (f) creates a variable whose domain is a set of strings.

In the previous example we used cell arrays to specify the elements of a domain.  When the domain consists only of numeric values (integer or floating-point), domains can instead be specified as a numeric array.  In this case, each element of the array (regardless of the array's dimensions) is considered a distinct entry in the domain.

\begin{lstlisting}
a = Discrete(0:2);
b = Discrete([1 2 3; 4 5 6]);
c = Discrete([0:2]');
\end{lstlisting}

(a) creates a variable with a domain of 0, 1, and 2.  (b) creates a variable with a domain of 1, 2, 3, 4, 5, 6.  (c) creates a variable with domain of 0, 1, and 2.

\fi

The domain may also be specified using a DiscreteDomain object.  In that case, the domain of the variable consists of the elements of this object.  For example:

\ifmatlab
\begin{lstlisting}
myDomain = DiscreteDomain(0:10);
a = Discrete(myDomain);
\end{lstlisting}
\fi

\ifjava
\begin{lstlisting}
DiscreteDomain myDomain = DiscreteDomain.create(0,1,2);
Discrete a = new Discrete(myDomain);
\end{lstlisting}
\fi

See section~\ref{sec:DiscreteDomain} for more information about the DiscreteDomain class.

\ifmatlab
\subpara{List of Matrix Dimensions}
\label{sec:VariableMatrixDimensions}

If the variable constructor is called without any dimensions, a single variable will be created.

If one dimension n is specified, a square array of dimensions n x n variables will be created\footnote{This follows a common MATLAB convention.}.

With k dimensions specified, n1, n2, ..., nk, a multidimensional variable array of dimensions n1 x n2 x ... x nk will be created.

\fi

\para{Properties}

\subpara{Domain}
\label{sec:Discrete.Domain}

Read-only.  The Domain property returns the domain of that variable in the form of a DiscreteDomain object.

\subpara{Belief}
\label{sec:Discrete.Belief}

Read-only.  For any single variable, the Belief method returns a vector whose length is the total number of elements of the domain of the variable.  When called after running a solver to perform inference on the graph, each element of the vector contains the estimated marginal probability of the corresponding element of the domain of the variable.  The results are undefined if called prior to running a solver.

For an array of variables, the Belief method will return an array of vectors (that is, an array one dimension larger than the variable array) containing the beliefs of each variable in the array.

\subpara{Value}
\label{sec:Discrete.Value}

Read-only.  In some cases, one may wish to retrieve the single most likely element of a variable's domain.  The Value property does just that.

For any single variable, the Value method returns a single value chosen from the domain of the variable.  When called after running a solver to perform inference on the graph, the value returned corresponds to the element in the variable's domain that has the largest estimated marginal probability\footnote{If more than one domain element has identical marginal probabilities that are larger than for any other value, a single value from the domain is returned, chosen arbitrarily among these.}.  The results are undefined if called prior to running a solver.

For an array of variables, the Value method will return an array of values, each from the domain of the corresponding variable representing the largest estimated marginal probability for that variable.

\subpara{Input}
\label{sec:Discrete.Input}

Read-write.  For any variable, the Input method can be used to set and return the current input of that variable. An input behaves much like a single edge factor connected to the variable, and is typically used the represent the likelihood function associated with a measured value (see section~\ref{sec:LikelihoodInput}).

When read, for a single variable returns an array of values with each value representing the current input setting for the corresponding element of the variable's domain.  The length of this array is equal to the total number of elements of the domain.  When read, for an array of variables, the result is an array with dimension one larger than the dimension of the variable array.  The additional dimension represents the current set of input values for the corresponding variable in the array.

When written, for a single variable, the value must be an array of length equal to the domain of the variable.  The values in the array must all be non-negative, and non-infinite, but are otherwise arbitrary.  When written, for an array of variables, the values must be a multidimensional array where the first set of dimensions exactly match the dimensions of the array (or the portion of the array) being set, and length of the last dimension is the number of elements in the variable's domain.


\subpara{FixedValue}
\label{sec:Discrete.FixedValue}

Read-write.  For any variable, the FixedValue property can be used to set the variable to a specific fixed value, and to retrieve the fixed-value if one has been set.  This would generally be used for conditioning a graph on known data without modifying the graph (see section~\ref{sec:FixingAVariableValue}).

Reading this property results in an error if no fixed value has been set.  To determine if a fixed value has been set, use the hasFixedValue method (see section~\ref{sec:Discrete.hasFixedValue}).

When setting this property on a single variable, the value must be a value included in the domain of the variable.
The fixed value must be a value chosen from the domain of the variable.  For example:

\ifmatlab
\begin{lstlisting}
a = Discrete(1:10);
a.FixedValue = 3;
\end{lstlisting}
\fi

\ifjava
\begin{lstlisting}
Discrete a = new Discrete(1,2,3);
a.setFixedValue(3);
\end{lstlisting}
\fi

When setting this property on a variable array, the value must be an array of the same dimensions as the variable array, and each entry in the array must be an element of the domain.

Because the Input and FixedValue properties serve similar purposes, setting one of these overrides any previous use of the other.  Setting the Input property removes any fixed value and setting the FixedValue property replaces the input with a delta function---the value 0 except in the position corresponding to the fixed value that had been set.


\para{Methods}

\subpara{hasFixedValue}
\label{sec:Discrete.hasFixedValue}

This method takes no arguments.  When called for a single variable, it returns a boolean indicating whether or not a fixed-value is currently set for this variable.  When called for a variable array, it returns a boolean array of dimensions equal to the size of the variable array, where each entry indicates whether a fixed value is set for the corresponding variable.

\subsubsection{Bit}

A Bit is a special kind of Discrete with domain [0 1].

\para{Constructor}


\ifmatlab
The Bit constructor can be used to create an N-dimensional collection of Dimple Discrete variables.  Its constructor does not require a domain, since the domain is predetermined.  The constructor takes only a variable-length list of matrix dimensions, where an empty list indicates a single Bit variable.

\begin{lstlisting}
Bit([dimensions])
\end{lstlisting}

The behavior of the list of dimensions is identical to that for Discrete variables as described in section~\ref{sec:VariableMatrixDimensions}.

\fi

\ifjava
\begin{lstlisting}
Bit()
\end{lstlisting}
\fi

\para{Properties}

\subpara{Domain}

See section~\ref{sec:Discrete.Domain}.

\subpara{Belief}

Read-only.  For a single Bit variable, the Belief property is a single number that represents the estimated marginal probability of the value one.

For an array of Bit variables, the Belief property is an array of numbers with size equal to the size of the variable array, with each value representing the estimated marginal probability of one for the corresponding variable.

\subpara{Value}

See section~\ref{sec:Discrete.Value}.

\subpara{Input}

Read-write.  For setting the Input property on a single Bit variable, the value must be a single number in the range 0 to 1, which represents the normalized likelihood of the value one (see section~\ref{sec:LikelihoodInput}).  If $L(x)$ is the likelihood of the variable, the Input should be set to $\frac{L(x=1)}{L(x=0) + L(x=1)}$.

For setting the Input property on an array of Bit variables, the value must be an array of normalized likelihood values, where the array dimensions must match the dimensions of the array (or the portion of the array) being set.


\subpara{FixedValue}

See section~\ref{sec:Discrete.FixedValue}.

\para{Methods}

\subpara{hasFixedValue}

See section~\ref{sec:Discrete.hasFixedValue}.


\subsubsection{Real}

A Real variable represents a variable that takes values on the real line, or on a contiguous subset of the real line.  \ifmatlab The Real class corresponds to either a single Real variable or a multidimensional array of Real variables.  All properties/methods can either be called for all elements in the collection or for individual elements of the collection.  \fi

\para{Constructor}

\ifmatlab
\begin{lstlisting}
Real([domain], [dimensions])
\end{lstlisting}
\fi

\ifjava
\begin{lstlisting}
Real()
Real(RealDomain)
Real(lowerBound,upperBound)
\end{lstlisting}
\fi

\ifmatlab
All arguments are optional and can be used in any combination.

\begin{itemize}
\item domain is specifies a bound on the domain of the variable. It can either be specified as a two-element array or a RealDomain object (see section~\ref{sec:RealDomain}).  If specified as an array, the first element is the lower bound and the second element is the upper bound. -Inf and Inf are allowed values for the lower or upper bound, respectively.  If no domain is specified, then a domain from $-\infty$ to $\infty$ is assumed.
\item dimensions specify the array dimensions.  The behavior of the list of dimensions is identical to that for Discrete variables as described in section~\ref{sec:VariableMatrixDimensions}.
\end{itemize}
\fi

\ifjava
\begin{itemize}
\item domain is specifies a bound on the domain of the variable. It can either be specified as two elements or a RealDomain object (see section~\ref{sec:RealDomain}).  If specified as two values, the first element is the lower bound and the second element is the upper bound. -Inf and Inf are allowed values for the lower or upper bound, respectively.  If no domain is specified, then a domain from $-\infty$ to $\infty$ is assumed.
\end{itemize}
\fi

Examples:

\ifmatlab
\begin{itemize}
\item Real() specifies a scalar real variable with an unbounded domain.
\item Real(4,1) specifies a 4x1 vector of real variables with unbounded domain.
\item Real([-1 1]) specifies a scalar real variable with domain from -1 to 1.
\item Real([-Inf 0], 4, 10, 2) specifies a 4x10x2 array of real variables, each with the domain from negative infinity to zero.
\item Real(RealDomain(-pi, pi)) specifies a scalar real variable with domain from $-\pi$ to $\pi$.
\end{itemize}
\fi

\ifjava
\begin{itemize}
\item Real() specifies a scalar real variable with an unbounded domain.
\item Real(-1, 1) specifies a scalar real variable with domain from -1 to 1.
\item Real(Double.NEGATIVE\_INFINITY,0) specifies a variable with the domain from negative infinity to zero.
\item Real(RealDomain.create(-Math.PI, Math.PI)) specifies a scalar real variable with domain from $-\pi$ to $\pi$.
\end{itemize}
\fi


\para{Properties}

\subpara{Domain}

Read-only.  The Domain property returns the domain of that variable in the form of a RealDomain object (see section~\ref{sec:RealDomain}).

\subpara{Belief}

Read-only.  The behavior of this property for Real variables is solver specific.  Some solvers do not support this property at all and will return an error when read.  See section~\ref{sec:SolversAPI} for more detail on each of the supported solvers.


\subpara{Value}

Read-only.  The behavior of this property for Real variables is solver specific.  Some solvers do not support this property at all and will return an error when read.  See section~\ref{sec:SolversAPI} for more detail on each of the supported solvers.

For the Gaussian solver, the Value corresponds to the mean value of the belief.


\subpara{Input}

Read-write.  For a Real variable, the form of the Input property depends on the particular solver.  For the Gibbs and ParticleBP solvers, the Input property must be a built-in factor function, as described in section~\ref{sec:usingBuiltInFactors}.  Typically, it would be one of the standard distributions included in the list of available built-in factor functions.  In this case, it must be one in which all the parameters can be fixed to pre-defined constants.  For example:

\ifmatlab
\begin{lstlisting}
r = Real();
r.Input = FactorFunction('Normal', measuredMean, measurementPrecision);
\end{lstlisting}
\fi

\ifjava
\begin{lstlisting}
Real r = new Real();
r.setInput(new Normal(measuredMean, measurementPrecision));
\end{lstlisting}
\fi

For the Gaussian solver, the Input property is a two-element array, where the first element is the mean, and the second element is the standard deviation.  For example:

\ifmatlab
\begin{lstlisting}
r = Real();
r.Input = [measuredMean, measurementStandardDeviation];
\end{lstlisting}
\fi

\ifjava
\begin{lstlisting}
Real r = new Real();
r.setInput( new double [] {measuredMean, measurementStandardDeviation});
\end{lstlisting}
\fi

In the current version of Dimple, Inputs on Real variable arrays must be set one at a time, or all set to a single common value\footnote{This restriction may be removed in a future version.}.

\subpara{FixedValue}

Read-write.  The behavior of the FixedValue property for a Real variable is nearly identical to that of Discrete variables (see section~\ref{sec:Discrete.FixedValue}).  When setting the FixedValue of a Real variable, the value must be within the domain of the variable, that is greater than or equal to the lower bound and less than or equal to the upper bound.  For example:

\ifmatlab
\begin{lstlisting}
a = Real([-pi pi]);
a.FixedValue = 1.7;
\end{lstlisting}
\fi

\ifjava
\begin{lstlisting}
Real a = new Real(-Math.PI,Math.PI);
a.setFixedValue(1.7);
\end{lstlisting}
\fi

Because the Input and FixedValue properties serve similar purposes, setting one of these overrides any previous use of the other.  Setting the Input property removes any fixed value and setting the FixedValue property removes the input.


\para{Methods}

\subpara{hasFixedValue}

See section~\ref{sec:Discrete.hasFixedValue}.



\subsubsection{RealJoint}

A RealJoint variable is a tightly coupled set of real variables that are treated by a solver as a single joint variable rather than a separate collection of variables.  For example, in the Gaussian solver, the messages associated with RealJoint variables involve joint mean and covariance matrix rather than an individual mean and variance for each variable.

Like other variables, the RealJoint class can represent either a single RealJoint variable (representing a collection of real values) or an array of RealJoint variables.

\para{Constructor}

\ifmatlab
\begin{lstlisting}
RealJoint(numElements, [dimensions])
\end{lstlisting}

The arguments are defined as follows:

\begin{itemize}
\item numElements specifies the number of joint real-valued elements.
\item dimensions specify the array dimensions (the array of individual RealJoint variables).  The behavior of the list of dimensions is identical to that for Discrete variables as described in section~\ref{sec:VariableMatrixDimensions}.
\end{itemize}
\fi

\ifjava
\begin{lstlisting}
RealJoint(int size)
RealJoint(RealJointDomain domain)
\end{lstlisting}

The arguments are defined as follows:

\begin{itemize}
\item size specifies the number of joint real-valued elements.
\item domain specifies the domain of the RealJoint variable using a RealJointDomain object (see \ref{sec:RealJointDomain}).  Using this version of the constructor allows bounds to be specified in some or all dimensions of the domain.
\end{itemize}
\fi

\para{Properties}

\subpara{Domain}
\label{sec:RealJoint.Domain}

Read-only.  The Domain property returns the domain of that variable in the form of a RealJointDomain object (see section~\ref{sec:RealJointDomain}).

\subpara{Belief}
\label{sec:RealJoint.Belief}

Read-only.  The behavior of this property for Real variables is solver specific.  Some solvers do not support this property at all and will return an error when read.  See section~\ref{sec:SolversAPI} for more detail on each of the supported solvers.

For the Gaussian solver, this property returns the estimated marginal distribution of the variable in the form of a MultivariateMsg (see section~\ref{sec:MultivariateMsg}), which includes a mean vector and covariance matrix.  \ifmatlab For an array of RealJoint variables, this property returns a cell array of MultivariateMsg objects, each corresponding to the estimated marginal distribution of the corresponding variable. \fi The results are undefined if called prior to running a solver.

\subpara{Value}
\label{sec:RealJoint.Value}

Read-only.  The behavior of this property for Real variables is solver specific.  Some solvers do not support this property at all and will return an error when read.  See section~\ref{sec:SolversAPI} for more detail on each of the supported solvers.

For the Gaussian solver, this property returns the mean vector, with dimension equal to the dimension of the RealJoint variable.

\ifmatlab
For an array of RealJoint variables, this property returns an array where the initial dimensions correspond to the dimensions of the variable array (or portion of the variable array), and the final dimension corresponds to the dimension of the RealJoint variable.
\fi

\subpara{Input}
\label{sec:RealJoint.Input}

Read-write.  The behavior of this property for Real variables is solver specific.  

For the Gaussian solver, \ifmatlab for a single RealJoint variable, \fi the value of this property must be a MultivariateMsg (see section~\ref{sec:MultivariateMsg}), which includes a mean vector and covariance matrix.  \ifmatlab For an array of RealJoint variables, the value of this property must be a cell array of MultivariateMsg objects, with dimensions equal to the dimension of the array (or the portion of the array) being set. \fi

\ifmatlab
For the Gibbs or ParticleBP solvers, for a single RealJoint variable, the value of this property must be a cell array of FactorFunction objects (see section~\ref{sec:FactorFunction}), one for each dimension of the RealJoint variable.  In the current version of Dimple, for these solvers, Inputs on Real variable arrays must be set one at a time, or all set to a single common value\footnote{This restriction may be removed in a future version.}.
\fi
\ifjava
For the Gibbs or ParticleBP solvers, the value of this property must be FactorFunction (see section~\ref{sec:FactorFunction}).  For an array of RealJoint variables, the value of this property must be an array of FactorFunctions, with dimensions equal to the dimensions of the array (or portion of the array) being set.
\fi

\subpara{FixedValue}
\label{sec:RealJoint.FixedValue}

Read-write.  The behavior of the FixedValue property for a RealJoint variable is similar to that of Discrete variables (see section~\ref{sec:Discrete.FixedValue}).  When setting the FixedValue of a Real variable, the value must be within the domain of the variable.  When setting a fixed value, the value must be in an array with dimension equal to the dimension of the RealVariable.  For example:

\ifmatlab
\begin{lstlisting}
a = RealJoint(4);
a.FixedValue = [1.7, 2.0, 0, -1.2];
\end{lstlisting}
\fi

\ifjava
\begin{lstlisting}
RealJoint a = new RealJoint(4);
a.setFixedValue(new double[]{1.7, 2.0, 0, -1.2});
\end{lstlisting}
\fi

\ifmatlab
For an array of RealJoint variables, the fixed values of all variables may be set together.  In this case, the initial dimensions of the input array must equal the dimensions of the variable array (or subset of the variable array) being set, while the final dimension must equal the dimension of the RealJoint variable.
\fi

Because the Input and FixedValue properties serve similar purposes, setting one of these overrides any previous use of the other.  Setting the Input property removes any fixed value and setting the FixedValue property removes the input.


\para{Methods}

\subpara{hasFixedValue}

See section~\ref{sec:Discrete.hasFixedValue}.



\subsubsection{Complex}

Complex is a special kind of RealJoint variable with exactly two joint elements.

\ifmatlab
The behavior of all properties and methods is identical to that of RealJoint variables, with the exception of a few methods (described below), which that refer directly to complex numerical values. 
\fi
\ifjava
The behavior of all properties and methods is identical to that of RealJoint variables.
\fi

\para{Constructor}

\ifmatlab
\begin{lstlisting}
Complex([dimensions])
\end{lstlisting}

The behavior of the list of dimensions is identical to that for Discrete variables as described in section~\ref{sec:VariableMatrixDimensions}.
\fi

\ifjava
\begin{lstlisting}
Complex()
Complex(ComplexDomain domain)
\end{lstlisting}

The arguments are defined as follows:

\begin{itemize}
\item domain specifies the domain of the Complex variable using a ComplexDomain object (see \ref{sec:ComplexDomain}).  Using this version of the constructor allows bounds to be specified in some or all dimensions of the domain.
\end{itemize}
\fi

\para{Properties}

\subpara{Domain}

Read-only.  See section~\ref{sec:RealJoint.Domain}.

\subpara{Belief}

Read-only.  See section~\ref{sec:RealJoint.Domain}.

\subpara{Value}

\ifmatlab
Read-only.  This method behaves similarly to the Value property for a RealJoint variable, except the value returned is a single complex number.  For an array of Complex variables, each entry in the returned array is a complex number.
\fi
\ifjava
Read-only.  See section~\ref{sec:RealJoint.Value}.
\fi

\subpara{Input}

Read-write.  See section~\ref{sec:RealJoint.Input}.

\subpara{FixedValue}

\ifmatlab
Read-write.  This method behaves similarly to the FixedValue property for a RealJoint variable, except that the fixed value is a single complex number.  For an array of Complex variables, each entry in the array should be a complex number.  For example:

a = Complex();
a.FixedValue = 5 + 1i*2;
\fi
\ifjava
Read-write.  See section~\ref{sec:RealJoint.FixedValue}.
\fi

\para{Methods}

\subpara{hasFixedValue}

See section~\ref{sec:Discrete.hasFixedValue}.


\subsubsection{FiniteFieldVariable}

Dimple supports a special variable type called a FiniteFieldVariable, which represent finite fields with $N=2^{n}$ elements. These fields find frequent use in error correcting codes.  These variables are used along with certain custom factors that are implemented more efficiently for sum-product belief propagation than the alternative using discrete variables and factors implemented directly.  See section~\ref{sec:finiteFields} for more information on how these variables are used.

The behavior of all properties and methods is identical to that of Discrete variables.

\para{Constructor}

\ifmatlab
\begin{lstlisting}
FiniteFieldVariable(primitivePolynomial, [dimensions])
\end{lstlisting}
\fi

\ifjava
\begin{lstlisting}
FiniteFieldVariable(primitivePolynomial)
\end{lstlisting}
\fi

The arguments are defined as follows:

\ifmatlab
\begin{itemize}
\item primitivePolynomial the primitive polynomial of the finite field.  The format of the primitive polynomial follows the same definition used by MATLAB in the \texttt{gf} function.  See the MATLAB help on the \texttt{gf} function for more detail.
\item dimensions specify the array dimensions (the array of individual FiniteFieldVariable variables).  The behavior of the list of dimensions is identical to that for Discrete variables as described in section~\ref{sec:VariableMatrixDimensions}.
\end{itemize}
\fi

\ifjava
\begin{itemize}
\item primitivePolynomial the primitive polynomial of the finite field.  The format of the primitive polynomial follows the same definition used by MATLAB in the \texttt{gf} function.  See the MATLAB help on the \texttt{gf} function for more detail.
\end{itemize}
\fi

\subsubsection{DiscreteDomain}
\label{sec:DiscreteDomain}

The DiscreteDomain class represents a domain with a finited fixed set of elements. It is the type of Domain used
by Discrete variables. DiscreteDomain objects are immutable.

\para{Construction}

\begin{lstlisting}
DiscreteDomain(elementList)
\end{lstlisting}

\ifmatlab
The elementList argument is either a cell array or array of domain elements.  Every entry of the array or cell array is considered an element of the domain, regardless of the number of dimensions it has.  For a cell array, each object in the cell array is considered an element of the domain regardless of the object type.  For a numeric array, every entry in the array must be numeric.
\fi

\ifjava
The elementList argument is either an array or array of domain elements.  Every entry of the array is considered an element of the domain, regardless of the number of dimensions it has.  For an array of Objects, each object in the array is considered an element of the domain regardless of the object type.  For a numeric array, every entry in the array must be numeric.
\fi

\para{Properties}

\subpara{Elements}

Read-only.  This property returns the set of elements in the discrete domain in the form of a one-dimensional \ifmatlab cell \fi array.




\subsubsection{RealDomain}
\label{sec:RealDomain}

The RealDomain class is used to refer to the domain of Real variables.

\para{Constructor}

\ifmatlab
\begin{lstlisting}
RealDomain([lowerBound, [upperBound] ])
\end{lstlisting}

\begin{itemize}
\item lowerBound indicates the lower bound of the domain.  The value must be a scalar numeric value.  It may be set to -Inf to indicate that there is no lower bound.  The default value is -Inf.
\item upperBound indicates the upper bound of the domain.  The value must be a scalar numeric value.  It may be set to Inf to indicate that there is no upper bound.  The default value is Inf.
\end{itemize}

\fi

\ifjava
\begin{lstlisting}
RealDomain(lowerBound,upperBound)
RealDomain()
RealDomain(double [] bounds)
\end{lstlisting}

\begin{itemize}
\item lowerBound indicates the lower bound of the domain.  The value must be a scalar numeric value.  It may be set to -Inf to indicate that there is no lower bound.  The default value is -Inf.
\item upperBound indicates the upper bound of the domain.  The value must be a scalar numeric value.  It may be set to Inf to indicate that there is no upper bound.  The default value is Inf.
\end{itemize}

\fi

\para{Properties}

\ifmatlab
\subpara{LB}
\fi
\ifjava
\subpara{LowerBound}
\fi

Read-only.  This property returns the value of the lower bound.  The default value is -Inf.

\ifmatlab
\subpara{UB}
\fi
\ifjava
\subpara{UpperBound}
\fi

Read-only.  This property returns the value of the upper bound.  The default value is Inf.



\subsubsection{RealJointDomain}
\label{sec:RealJointDomain}

The RealJointDomain class is used to refer to the domain of RealJoint variables.

\para{Constructor}

\ifmatlab
\begin{lstlisting}
RealJointDomain([numDimensions], [listOfRealDomains]);
\end{lstlisting}

\begin{itemize}
\item numDimensions indicates the number of dimensions in the domain of the RealJoint variable.  May be omitted only if the listOfRealDomains arguments are present.  In that case, the length of that list determines the dimension.
\item listOfRealDomains comma-separated list of RealDomain objects, one for each dimension.  Each RealDomain in the list indicates the domain for the corresponding dimension of the RealJoint variable.  If no listOfRealDomains is specified, then all dimensions are assumed to be unbounded.
\end{itemize}
\fi

\ifjava
\begin{lstlisting}
RealJointDomain(int size)
RealJointDomain(RealDomain... domains)
\end{lstlisting}

\begin{itemize}
\item size indicates the number of dimensions in the domain of the RealJoint variable.  If the version of the constructor that specifies only the size is called, then all dimensions are assumed to be unbounded.
\item domains is a list or array of RealDomain objects, one for each dimension.  Each RealDomain in the list indicates the domain of the corresponding dimension of the RealJoint variable.  The number of entries determines the number of dimensions.
\end{itemize}
\fi

\ifmatlab
\para{Properties}

\subpara{NumElements}

Read-only.  Indicates the number of elements in the RealJointDomain, which corresponds to the number of dimensions of an associated RealJoint variable.
\fi

\ifjava
\para{Methods}

\subpara{getNumVars}

\begin{lstlisting}
domain.getNumVars();
\end{lstlisting}

Returns the number of elements in the RealJointDomain, which corresponds to the number of dimensions of an associated RealJoint variable.

\subpara{getRealDomains}

\begin{lstlisting}
domain.getRealDomains();
\end{lstlisting}

Returns the collection of RealDomains that correspond to each dimension of the RealJointDomain.

\subpara{getRealDomain}

\begin{lstlisting}
domain.getRealDomain(dimensionIndex);
\end{lstlisting}

Returns the RealDomain that correspond to the dimension corresponding to the specified \texttt{dimensionIndex}.

\fi


\subsubsection{ComplexDomain}
\label{sec:ComplexDomain}

The ComplexDomain class, a subclass of the RealJointDomain class, is used to refer to the domain of Complex variables.

\para{Constructor}

\ifmatlab
\begin{lstlisting}
ComplexDomain([listOfRealDomains]);
\end{lstlisting}

\begin{itemize}
\item listOfRealDomains comma-separated list of exactly two RealDomain objects, one for each dimension.  Each RealDomain in the list indicates the domain for the corresponding dimension of the Complex variable (real followed by imaginary).  If no listOfRealDomains is specified, then both dimensions are assumed to be unbounded.
\end{itemize}
\fi

\ifjava
\begin{lstlisting}
ComplexDomain()
ComplexDomain(RealDomain... domains)
\end{lstlisting}

\begin{itemize}
\item domains is a list or array of RealDomain objects, one for each dimension.  Each RealDomain in the list indicates the domain of the corresponding dimension of the Complex variable (real followed by imaginary).  If no domains argument is specified, then both dimensions are assumed to be unbounded.
\end{itemize}
\fi

\ifmatlab
\para{Properties}

\subpara{NumElements}

Read-only.  Indicates the number of elements in the ComplexDomain, which should always equal two.
\fi

\ifjava
\para{Methods}

\subpara{getNumVars}

\begin{lstlisting}
domain.getNumVars();
\end{lstlisting}

Returns the number of elements in the ComplexDomain, which should always equal two.

\subpara{getRealDomains}

\begin{lstlisting}
domain.getRealDomains();
\end{lstlisting}

Returns the collection of RealDomains that correspond to each dimension of the ComplexDomain (real followed by imaginary).

\subpara{getRealDomain}

\begin{lstlisting}
domain.getRealDomain(dimensionIndex);
\end{lstlisting}

Returns the RealDomain that correspond to the dimension corresponding to the specified \texttt{dimensionIndex}.

\fi



\subsubsection{MultivariateMsg}
\label{sec:MultivariateMsg}

The MultivariateMsg class is used to specify the parameters of a multivariate Gaussian distribution, as used in the Gaussian solver.

\para{Constructor}

\begin{lstlisting}
MultivariateMsg(meanVector, covarianceMatrix)
\end{lstlisting}

\begin{itemize}
\item meanVector indicates the mean value of each element in a joint set of variables.  The value must be a one-dimensional numeric array.
\item covarianceMatrix indicates the covariance matrix associated with the elements of a joint set of variables.  The value must be a two-dimensional numeric array with each dimension identical to the length of the meanVector.
\end{itemize}


\para{Properties}

\subpara{Means}

Read-only.  Returns a vector of values, where each value indicates the mean value of each element in a joint set of variables.

\subpara{Covariance}

Read-only.  Returns a two-dimensional array of values, representing the covariance matrix of a joint set of variables.







