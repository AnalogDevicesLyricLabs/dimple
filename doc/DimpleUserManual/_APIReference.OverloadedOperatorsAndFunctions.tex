\subsection{List of Overloaded MATLAB Operators and Functions}
\label{sec:overloaded}

The following table lists the set of overloaded MATLAB operators that can be used to implicitly create factors.  The table shows the operator, the corresponding built-in factor (as described in section~\ref{sec:builtInFactors}), the valid variable data types of the inputs and outputs (B~=~Bit, D~=~Discrete, or R~=~Real, C~=~Complex, RJ~=~RealJoint), and wether or not vectorized inputs are supported.  The use of these operators and functions is described in section~\ref{sec:ImplicitFactorCreation}.

\begin{longtable} {p{1.7cm} p{3.2cm} p{1cm} p{1cm} p{1cm} p{1.5cm} p{4.7cm}}
Operator & Factor & Out & In1 & In2 & Vectorized & Description \\
\hline
\endhead
%
$\&$ & And & B & B & B & \checkmark & Logical AND \\
$|$ & Or & B & B & B & \checkmark & Logical OR \\
xor() & Xor & B & B & B & \checkmark & Logical XOR \\
$\sim$ & Not & B & B & - & \checkmark & Logical NOT \\
$+$ & Sum & D,R\footnote{\label{ftn:outReal}If either input is Real, then the output is Real} & D,R & D,R & \checkmark & Plus \\*
 & ComplexSum & C & C,R & C,R & \checkmark & Complex plus \\
 & RealJointSum & RJ & RJ & RJ &  & RealJoint plus \\
$-$ & Subtract & D,R\textsuperscript{\ref{ftn:outReal}} & D,R & D,R & \checkmark & Minus \\*
 & ComplexSubtract & C & C,R & C,R & \checkmark & Complex minus \\*
 & RealJointSubtract & RJ & RJ & RJ &  & RealJoint minus \\*
 & Negate & D,R\textsuperscript{\ref{ftn:outReal}} & D,R & - & \checkmark & Unary minus \\*
 & ComplexNegate & C & C, R & - & \checkmark & Unary complex minus \\
 & RealJointNegate & RJ & RJ & - &  & Unary RealJoint minus \\
$*$ & Product & D,R\textsuperscript{\ref{ftn:outReal}} & D,R & D,R & \checkmark\footnote{\label{ftn:inScalar}One of the inputs may be a vector as long as the other is a scalar.} & Scalar multiply \\*
 & ComplexProduct & C & C,R & C,R & \checkmark\textsuperscript{\ref{ftn:inScalar}} & Complex scalar multiply \\*
 & VectorInnerProduct & R & D,R,RJ & D,R,RJ & - & Vector inner product\footnote{If both inputs are vectors of the same dimension, then the VectorInnerProduct factor will be used. Each vector may be an array of scalar variables or a RealJoint variable.} \\*
 & MatrixProduct & R & D,R & D,R & - & Matrix multiply\footnote{If both inputs are two-dimensional matrices of appropriate dimension, then the MatrixProduct factor will be used.} \\*
 & MatrixVectorProduct & R & D,R & D,R & - & Matrix-vector multiply\footnote{If one input is a vector and the other is a matrix of appropriate dimension, then the MatrixVectorProduct factor will be used.} \\
 & MatrixRealJoint\newline VectorProduct & RJ & RJ,D,R & RJ,D,R & - & Matrix-vector multiply\footnote{If one input is a RealJoint variable and the other is a matrix of scalar variables or constants of appropriate dimension, then the MatrixRealJointVectorProduct factor will be used.} \\
$.*$ & Product & D,R\textsuperscript{\ref{ftn:outReal}} & D,R & D,R & \checkmark & Point-wise multiply \\*
 & ComplexProduct & C & C,R & C,R & \checkmark & Complex pointwise multiply \\
$/$ & Divide & D,R\textsuperscript{\ref{ftn:outReal}} & D,R & D,R & \checkmark\footnote{\label{ftn:divScalar}The dividend may be a vector as long as the divisor is a scalar.} & Scalar divide \\*
 & ComplexDivide & C & C,R & C,R & \checkmark\textsuperscript{\ref{ftn:divScalar}} & Complex scalar divide \\
$./$ & Divide & D,R\textsuperscript{\ref{ftn:outReal}} & D,R & D,R & \checkmark & Point-wise divide \\*
 & ComplexDivide & C & C,R & C,R & \checkmark & Complex pointwise divide \\
$\wedge$ & Power & D,R\textsuperscript{\ref{ftn:outReal}} & D,R & D,R & \checkmark\footnote{The base may be a vector as long as the exponent is a scalar.} & Scalar power \\
$.\wedge$ & Power & D,R\textsuperscript{\ref{ftn:outReal}} & D,R & D,R & \checkmark & Point-wise power \\
$'$ & ComplexConjugate & C & C & - & \checkmark & Complex conjugate \\
$<$ & LessThan & B & D,R & D,R & \checkmark & Less than \\
$>$ & GreaterThan & B & D,R & D,R & \checkmark & Greater than \\
$<=$ & GreaterThan\footnote{Uses GreaterThan factor, reversing the order.} & B & D,R & D,R & \checkmark & Less than or equal to \\
$>=$ & LessThan\footnote{Uses LessThan factor, reversing the order.} & B & D,R & D,R & \checkmark & Greater than or equal to \\
Equals() & Equals & B & B,D,R & B,D,R\footnote{\label{ftn:equals}This function is not limited to two inputs, but can take an arbitrary number of inputs} & \checkmark & Equals\footnote{Equivalent to the $==$ operator, but the $==$ operator is not overloaded for this purpose so that it can instead be used to determine whether or not two variables reference the same Dimple variable.} \\
NotEquals() & NotEquals & B & B,D,R & B,D,R$^{\ref{ftn:equals}}$ & \checkmark & Not equals\footnote{Equivalent to the $\sim=$ operator, but the $\sim=$ operator is not overloaded for this purpose so that it can instead be used to determine whether or not two variables reference the same Dimple variable.} \\
mod() & - & D  & D & D & \checkmark & Modulo function\footnote{Currently, the mod() operator supports discrete variables only, and it uses the MATLAB definition of mod on negative numbers.  This may be subject to change in future versions.} \\
abs() & Abs & D,R\textsuperscript{\ref{ftn:outReal}} & D,R & - & \checkmark & Absolute value \\
sqrt() & Sqrt & R & R & - & \checkmark & Square root \\
log() & Log & R & R & - & \checkmark & Natural log \\
exp() & Exp & R & R & - & \checkmark & Exponential function \\*
 & ComplexExp & C & C & - & \checkmark & Complex exponential \\
sin() & Sin & R & R & - & \checkmark & Sine \\
cos() & Cos & R & R & - & \checkmark & Cosine \\
tan() & Tan & R & R & - & \checkmark & Tangent \\
asin() & ASin & R & R & - & \checkmark & Arc-sine \\
acos() & ACos & R & R & - & \checkmark & Arc-cosine \\
atan() & ATan & R & R & - & \checkmark & Arc-tangent \\
sinh() & Sinh & R & R & - & \checkmark & Hyperbolic sine \\
cosh() & Cosh & R & R & - & \checkmark & Hyperbolic cosine \\
tanh() & Tanh & R & R & - & \checkmark & Hyperbolic tangent \\
\end{longtable}



