\subsection{Graph Libraries}

Dimple provides a few graphs that are useful as nested graphs.

\subsubsection{Multiplexer CPDs}
\label{sec:multiplexerCPD}

Suppose you wanted a factor representing a DAG with the following probability distribution:

\[
p(Y=y|a,z_1,z_2,...) \propto \delta(y = z_a)
\]

where all variables are Discrete.

You could code this up in Dimple as follows:

\ifmatlab
\begin{lstlisting}
function weight = myFunc(y,a,z)
    weight = y == z(a);
end

N = 3; %Number of possible sources
M = 2; %Domain of Zs and Y

y = Discrete(1:M);
a = Discrete(1:N);
z = Discrete(1:M,N,1);
 
fg = FactorGraph();
 
fg.addFactor(@myFunc,y,a,z);
\end{lstlisting}
\fi

\ifjava
\begin{lstlisting}
public class Main 
{		
	public static class MyFunc extends FactorFunction
	{
		@Override
		public double eval(Object ... args)
		{
			Object y = args[0];
			int a = (Integer)args[1];
			
			Object z = args[2+a];
			
			return y.equals(z) ? 1.0 : 0.0;
		}
	}
	
	public static void main(String [] args)
	{
		
		Discrete y = new Discrete(1,2);
		Discrete a = new Discrete(0,1,2);
		Discrete z1 = new Discrete(1,2);
		Discrete z2 = new Discrete(1,2);
		Discrete z3 = new Discrete(1,2);
		 
		FactorGraph fg = new FactorGraph();
		 
		fg.addFactor(new MyFunc(),y,a,z1,z2,z3);
	}
}
\end{lstlisting}
\fi

However, to build this FactorTable takes $O(NM^{N+1})$ where N is the number of Zs and M is the domain size of the Zs.  Runtime is almost as bad at $O(NM^N)$.  However, there is an optimization that can result in $O(MN^2)$ runtime and graph building time.  Dimple provides a MultiplexerCPD graph that can be used as a nested graph to achieve this optimization.

\ifmatlab
\begin{lstlisting}
cpd = MultiplexerCPD({1,2},3);
Y = Discrete(1:2);
A = Discrete(1:3);
Z1 = Discrete(1:2);
Z2 = Discrete(1:2);
Z3 = Discrete(1:2);
fg = new FactorGraph();
fg.addFactor(cpd,Y,A,Z1,Z2,Z3);
\end{lstlisting}
\fi


\ifjava
\begin{lstlisting}
MultiplexerCPD cpd = new MultiplexerCPD(new Object [] {1,2},3);
Discrete Y = new Discrete(1,2);
Discrete A = new Discrete(0,1,2);
Discrete Z1 = new Discrete(1,2);
Discrete Z2 = new Discrete(1,2);
Discrete Z3 = new Discrete(1,2);

FactorGraph fg = new FactorGraph();
fg.addFactor(cpd,Y,A,Z1,Z2,Z3);
\end{lstlisting}
\fi

Dimple supports each Z having different domains.  In this case, Y's domain must be the sorted union of all the Z domains

\ifmatlab
\begin{lstlisting}
cpd = MultiplexerCPD({{1,2},{1,2,3},{2,4}});
Y = Discrete(1:4);
A = Discrete(1:3);
Z1 = Discrete(1:2);
Z2 = Discrete(1:3);
Z3 = Discrete([2 4]);
fg = FactorGraph();
fg.addFactor(cpd,Y,A,Z1,Z2,Z3);
\end{lstlisting}
\fi

\ifjava
\begin{lstlisting}
Object [][] domains = new Object [][] {
	new Object [] {1,2},
	new Object [] {1,2,3},
	new Object [] {2,4}
};

MultiplexerCPD cpd = new MultiplexerCPD(domains);
Discrete Y = new Discrete(1,2,3,4);
Discrete A = new Discrete(0,1,2);
Discrete Z1 = new Discrete(1,2);
Discrete Z2 = new Discrete(1,2,3);
Discrete Z3 = new Discrete(2,4);

FactorGraph fg = new FactorGraph();
fg.addFactor(cpd,Y,A,Z1,Z2,Z3);
\end{lstlisting}
\fi

Note that when using the SumProduct solver, a custom implementation of the built-in `Multiplexer' factor function (see section~\ref{sec:builtInFactors}) exists that is even more efficient than using the MultiplexerCPD graph library function.  When using other solvers with discrete variables, the MultiplexerCPD graph library function should be more efficient.  When the Z variables are real rather than discrete, the `Multiplexer' factor function is the only option.


\ifmatlab
\subsubsection{N-Bit Xor Definition}
An N-bit soft xor can be decomposed into a tree of three bit soft xors.  Dimple provides the getNBitXorDef function to generate such a graph.  The following code shows how to use such a graph.

\begin{lstlisting}
fg = FactorGraph();
fg.addFactor(getNBitXorDef(4),Bit(4,1));
\end{lstlisting}

\fi