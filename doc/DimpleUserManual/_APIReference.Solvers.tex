\subsection{Solvers}
\label{sec:SolversAPI}

% TODO: make a separate file for each solver

\subsubsection{Solver-Specific Options}

Each solver supports a number of options specific to that solver. These are described in the following sections. Solver-specific options may be used to configure how overall inference works for that solver or may be used to configure the behavior of individual factors or variables for that solver. Solver-specific options are typically set on the applicable model object (factor graph, factor or variable) and the values will be observed and used to configure the corresponding solver objects:

\ifmatlab
\begin{lstlisting}
%*\textit{model-object}*.setOption('%*\textit{SolverOptionClass.optionName}*', %*\textit{option-value}*)
\end{lstlisting}
\fi
\ifjava
\begin{lstlisting}
%*\textit{model-object}*.setOption(%*\textit{SolverOptionClass.optionName}*, %*\textit{option-value}*)
\end{lstlisting}
\fi

Options that are not applicable to the object on which it is set or that are not applicable to the active solver will simply be ignored. For more details on the options mechanism see \autoref{sec:Options} of this document.

\subsubsection{Solver-Specific Methods}

Each solver also may support solver-specific methods, which are described in the following sections. As with options, solver-specific methods may be available for various objects: a factor-graph, variable, or factor. In each case, to call a solver-specific method, the method is applied to the solver object, returned by the Solver property.  For example:

\ifmatlab
\begin{lstlisting}
factorGraph.Solver.%*\textit{solverSpecificMethod}*(arguments);
\end{lstlisting}

\begin{lstlisting}
variable.Solver.%*\textit{solverSpecificMethod}*(arguments);
\end{lstlisting}

\begin{lstlisting}
factor.Solver.%*\textit{solverSpecificMethod}*(arguments);
\end{lstlisting}
\fi

\ifjava
\begin{lstlisting}
((%*\textit{SolverSpecificGraph}*)fg.getSolver()).%*\textit{solverSpecificMethod}*(arguments);
\end{lstlisting}

\begin{lstlisting}
((%*\textit{SolverSpecificVar}*)variable.getSolver()).%*\textit{solverSpecificMethod}*(arguments);
\end{lstlisting}

\begin{lstlisting}
((%*\textit{SolverSpecificFactor}*)factor.getSolver()).%*\textit{solverSpecificMethod}*(arguments);
\end{lstlisting}

The downcast in the first cast can be avoided if you save the return value from \lstname{setSolverFactory} and call the methods through that. For example:

\begin{lstlisting}
JunctionTreeSolverGraph sgraph =
    fg.setSolverFactory(new JunctionTreeSolver());
sgraph.useConditioning(true);
\end{lstlisting}

The downcasts are also not required for common methods available to all solvers described in the next section.
\fi

Some solver-specific methods return results, while others do not.  Some solver-specific methods require arguments, while others do not.  \ifmatlab If no arguments are needed, the parentheses are optional. \fi

\ifmatlab

In some cases it is convenient to call solver-specific methods on each element in an array of objects.  Utility methods are provided for this purpose.  Specifically, to call a solver-specific method that has no return value:
\begin{lstlisting}
objectArray.invokeSolverSpecificMethod('methodName', arguments);
\end{lstlisting}

In this case, 'methodName' is a text string with the name of the solver-specific method, and arguments is an optional comma-separated list of arguments to that method.

To call a solver specific method that has a return value:
\begin{lstlisting}
returnArray = objectArray.invokeSolverSpecificMethodWithReturnValue('methodName', arguments);
\end{lstlisting}

In this case, returnArray is a cell-array of the return values of the method, with dimensions equal to the dimensions of the object array.

\fi

\subsubsection{Common Options}

A few options are applicable to multiple solvers and are therefore described in this subsection.

\para{DimpleOptions.randomSeed}

\dimpleOption{DimpleOptions.randomSeed}
{64-bit integer}
{N/A}
{graph}
{When set, this option specifies a random seed that may be used by solvers that use a random number generator. The seed will only be used if explicitly set; the default value is not used. This can be used to ensure repeatable behavior during testing or profiling but should not be used for normal operation.}

\para{SolverOptions.enableMultithreading}

\dimpleOption{SolverOptions.enableMultithreading}
{boolean}
{false}
{graph}
{Controls whether to use multithreading for this solver. Multithreading is currently only supported by the MinSum and SumProduct solvers but will eventually be implemented in others. This value will be ignored if not applicable.}

\para{SolverOptions.iterations}

\dimpleOption{SolverOptions.iterations}
{integer}
{1}
{graph}
{Controls how many iterations to perform during solves. This is not applicable to all solvers. It is currently only used by Sum-Product, Min-Sum and Particle BP.}

\subsubsection{Common Methods}

There are also some methods that are common to all solvers. These are:

\para{getMultithreadingManager}

Dimple users can retrieve a MultithreadingManager on which to perform additional actions.

\ifmatlab
\begin{lstlisting}
fg.Solver.getMultithreadingManager()
\end{lstlisting}
\fi

\ifjava
\begin{lstlisting}
fg.getSolver().getMultithreadingManager()
\end{lstlisting}
\fi

Users can configure both the multithreading mode and the number of workers using the MultithreadingManager.

\subparagraph{Multithreading Modes}

Dimple provides various multithreading algorithms that have different speed advantages depending on the size of the user's graph and FactorTables.  In the future Dimple should be modified to automatically detect the best threading algorithm.  Currently, however, it defaults to the "Phase" multithreading mode and requires the user manually set the mode to change this.  For a given graph, users can try both modes and see which is faster.

The currently supported multithreading modes are:

\begin{itemize}
\item Phase - Divides the schedule into "phases" where each phase contains schedule entries that are entirely independent of one another.  These phases are then easy to parallelize.  
\item SingleQueue - Uses a single queue and a dependency graph to pull off work for each thread on the fly.  
\end{itemize}

The following methods can be used for getting and setting modes:


\ifmatlab

\begin{itemize}
\item fg.Solver.getMultithreadingManager().getModes() - Returns a Java array of enums specifying the valid modes.
\item fg.Solver.getMultithreadingManager().setMode(ModeName) - Allows users to set the mode by string.  Currently "Phase" or "SingleQueue" will work.
\item fg.Solver.getMultithreadingManager().setMode(enum) - Allows users to set the mode by the enums returned by the getModes method.
\end{itemize}

\fi

\ifjava
\begin{itemize}
\item fg.getSolver().getMultithreadingManager().getModes() - Returns an array of enums specifying the valid modes.
\item fg.getSolver().getMultithreadingManager().setMode(ModeName) - Allows users to set the mode by string.  Currently "Phase" or "SingleQueue" will work.
\item fg.getSolver().getMultithreadingManager().setMode(enum) - Allows users to set the mode by the enums returned by the getModes method or with MultithreadingMode.<PhaseName>.
\end{itemize}
\fi

\subparagraph{Setting Number of Threads and Workers}

Dimple provides a ThreadingPool as a singleton for multithreading.  It sets the number of threads in this pool to the number of virtual cores in the user's machine by default.  Users can override this default value.  In addition, Dimple allows users to specify the number of "workers" for a given FactorGraph.  This "NumWorkers" is also set to the number of virtual cores on the user's machine by default.  Whereas NumThreads specifies how many threads are in the threadPool, NumWorkers specifies how work is divided up across the graph.  These workers are run by the thread pool.  Best performance is achieved when NumWorkers and NumThreads are the same.  However, NumThreads is global and shared by all graphs where NumWorkers is specific to a given FactorGraph.

The following methods can be used to change number of workers:

\ifjava
\begin{itemize}
\item fg.getSolver().getMultithreadingManager().getNumWorkers()
\item fg.getSolver().getMultithreadingManager().setNumWorkers(num)
\item fg.getSolver().getMultithreadingManager().setNumWorkersToDefault()
\end{itemize}
\fi

\ifmatlab
\begin{itemize}
\item fg.Solver.getMultithreadingManager().getNumWorkers()
\item fg.Solver.getMultithreadingManager().setNumWorkers(num)
\item fg.Solver.getMultithreadingManager().setNumWorkersToDefault()
\end{itemize}
\fi

The following global methods can be used to set the number of threads in the ThreadPool

\ifmatlab
\begin{itemize}
\item getDimpleNumThreads()
\item setDimpleNumThreads(numThreads)
\item setDimpleNumThreadsToDefault()
\end{itemize}
\fi

\ifjava
\begin{itemize}
\item ThreadPool.getNumThreads()
\item ThreadPool.setNumThreads(numThreads)
\item ThreadPool.setNumThreadsToDefault()
\end{itemize}
\fi

\clearpage
\subsubsection{Sum-Product Solver}
\label{sec:SumProductSolver}

Use of the sum-product solver is specified by calling:

\ifmatlab
\begin{lstlisting}
fg.Solver = 'SumProduct';
\end{lstlisting}
\fi

\ifjava
\begin{lstlisting}
SumProductSolverGraph sfg = fg.setSolverFactory(new SumProductSolver());
\end{lstlisting}
\fi

If no solver is specified, the SumProduct solver is used by default.

The SumProduct solver supports both discrete and continuous variables.  The SumProduct solver uses the sum-product form of belief propagation to perform inference on a graph.  For discrete variables, each message to or from a factor is in the form of a vector of length equal to the domain size of the variable.  For continuous variables, messages are represented using a Gaussian parameterization.  In some cases, this is an approximation to the exact message.  For Real variables, a message is in the form of a pair of values representing the mean and variance of the corresponding Normal distribution.  For Complex and RealJoint variables, a message is in the form of a vector and matrix, representing the mean and covariance of the corresponding multivariate Normal distribution.

While the Gaussian representation of messages for continuous variables is sometimes an approximation, there are some specific built-in factors for which exact Gaussian messages are computed.  This can be done when a factor preserves the Gaussian form of the distribution on each edge.  The following table is a list of such lists built-in factors.  See section~\ref{sec:builtInFactors} for more information on built-in factors.

\begin{longtable} {p{3.5cm} p{3.0cm} p{7.0cm}}
Built-in Factor & Variable Types &  Notes \\
\hline
\endhead
Normal & Real & Applies only if mean and precision parameters are constants and all connected variables are Real and unbounded\footnote{Unbounded means that the domain of the variable must not have finite upper or lower bounds}. \\
MultivariateNormal & Complex or \newline RealJoint & All connected variables must be RealJoint or Complex and unbounded. \\
Sum & Real & All connected variables must be Real and unbounded.  \\
Subtract & Real & All connected variables must be Real and unbounded. \\
Negate & Real & All connected variables must be Real and unbounded. \\
ComplexSum & Complex & All connected variables must be Complex and unbounded. \\
ComplexSubtract & Complex & All connected variables must be Complex and unbounded. \\
ComplexNegate & Complex & All connected variables must be Complex and unbounded. \\
RealJointSum & RealJoint & All connected variables must be RealJoint of the same dimension and unbounded. \\
RealJointSubtract & RealJoint & All connected variables must be RealJoint of the same dimension and unbounded. \\
RealJointNegate & RealJoint & All connected variables must be RealJoint of the same dimension and unbounded. \\
Product & Real, Constant & Applies only if the product is one unbounded Real variable times one scalar constant to produce an unbounded Real variable. \\
MatrixRealJoint\newline VectorProduct & RealJoint, \newline Constant & Applies only if the product is one unbounded RealJoint variable times a constant matrix to produce an unbounded RealJoint variable. \\
LinearEquation & Real & Linear equation.  All connected variables must be Real and unbounded. \\
\end{longtable}

For factors that are neither discrete-only or listed in the above table, an approximate computation is used in computing messages from such a factor.  This includes any factor that connects to both discrete and continuous variables as well as factors that connect only to continuous variables but do not appear in the list above.  The approximate method is sample based and uses Gibbs sampling to sample from the factor, allowing approximate messages to be computed from the sample statistics.  Several methods described below allow control over the behavior of these sampled factors.

For discrete-only factors, two factor update algorithms are available: normal and optimized. The optimized algorithm can be applied only to factors with more than one edge, and only when the schedule updates all of the factor's edges simultaneously. The optimized algorithm computes the outbound message update with fewer operations than the normal algorithm, which can decrease execution time; however, it also uses more memory and increases initialization time. Several options, described below, influence which algorithm is used. Key among them is the updateApproach option, which can be set to normal, optimized, or automatic. When set to automatic, Dimple makes an estimate of the memory usage and execution time of each algorithm in order to select one.

\para{Sum-Product Options}
\label{sec:SumProductOptions}

The following options can be used with the SumProduct solver.

\subpara{SumProductOptions.damping}

\dimpleOption{SumProductOptions.damping}
{double}
{0.0}
{variables and factors}
{The SumProduct solver supports damping, in which messages are damped by replacing each message by a weighted sum of the computed message value and the previous value of that message (when the corresponding edge was most-recently updated). In the current version of Dimple, damping is supported only in discrete variables and factors that connect only to discrete variables\footnote{Support for damping for continuous variables may be added in a future release.}.
\linebreak
\linebreak
The damping parameter specifies a weighting value in the range 0 through 1:

\[
\mathrm{message} = \mathrm{computedMessage} \cdot (1 - D) + \mathrm{previousMessage} \cdot D
\]

where $D$ is the damping value. So that a value of 0 means that the previous message will not be considered, effectively turning off damping.
\linebreak
\linebreak
This option applies the same damping parameter to all edges connected to the variable or factor on which it is set. If you want different values for different edges, you need to use the \nameref{option:SumProductOptions.nodeSpecificDamping} option.
}

\subpara{SumProductOptions.maxMessageSize}

\dimpleOption{SumProductOptions.maxMessageSize}
{integer}
{integer max}
{discrete factors}
{This specifies the maximum size of the outgoing messages on the discrete factors on which it is set. If this number K is less than the full size of the message (i.e. the size of the domain of the target variable), then only the K-best values -- those with the largest weights -- will be included in the message. This can results in a faster but more approximate form of inference and is most suited to graphs with very large-dimension variables.
\linebreak
\linebreak
IMPORTANT: k-best and damping are not compatible with each other\footnote{This restriction may be removed in a future version of Dimple.}}

\subpara{SumProductOptions.nodeSpecificDamping}

\dimpleOption{SumProductOptions.nodeSpecificDamping}
{\ifmatlab double vector\fi \ifjava OptionDoubleList\fi}
{\textit{empty}}
{variables and factors}
{This is the similar to the \nameref{option:SumProductOptions.damping} option but allows you specify different weights for different edges. Unlike the simple damping option, this usually makes no sense to set on the graph itself since factors and variables will typically have different numbers and arrangements of edges. The value must either be an empty list, indicating that damping should be turned off, or a list of weights with the same length as the number of siblings of the affected variable and factor. The dmpaing weights will be applied in the order in which the siblings are declared.
\linebreak
\linebreak
This option takes precedence over the simple damping option if both are specified for the same node.
}

\subpara{SumProductOptions.updateApproach}

\dimpleOption{SumProductOptions.updateApproach}
{\ifmatlab string\fi \ifjava UpdateApproach enum\fi}
{AUTOMATIC}
{discrete factors}
{This option controls which update algorithm is applied to discrete factors. The option can take one of three values:
\begin{itemize}
\item NORMAL - Perform updates using just the factor tables. Do not use the optimized update technique.
\item OPTIMIZED - Use the optimized update algorithm. Note that factors that have only one edge, or factors that do not have all of their edges updated simultaneously by the schedule, ignore this setting and use the normal approach.
\item AUTOMATIC - Automatically determine whether to use the optimized algorithm. The automatic selection algorithm can be tuned through the \nameref{option:SumProductOptions.automaticExecutionTimeScalingFactor} and \nameref{option:SumProductOptions.automaticMemoryAllocationScalingFactor} options.
\end{itemize}
}

\subpara{SumProductOptions.automaticExecutionTimeScalingFactor}

\dimpleOption{SumProductOptions.automaticExecutionTimeScalingFactor}
{double}
{1.0}
{discrete factors}
{This option is an execution time scaling factor used when the \nameref{option:SumProductOptions.updateApproach} option is set to AUTOMATIC. It controls how execution time costs are weighed. The value must be a positive number.}

\subpara{SumProductOptions.automaticMemoryAllocationScalingFactor}

\dimpleOption{SumProductOptions.automaticMemoryAllocationScalingFactor}
{double}
{10.0}
{discrete factors}
{This option is an memory allocation scaling factor used when the \nameref{option:SumProductOptions.updateApproach} option is set to AUTOMATIC. It controls how memory allocation costs are weighed. The value must be a positive number.}

\subpara{SumProductOptions.optimizedUpdateSparseThreshold}

\dimpleOption{SumProductOptions.optimizedUpdateSparseThreshold}
{double}
{1.0}
{discrete factors}
{This option controls the representation of the auxiliary tables used by the optimized update algorithm, which is controlled through the \nameref{option:SumProductOptions.updateApproach}. Internally, the optimized algorithm creates multiple factor tables to perform the update. This option specifies a density, below which an auxiliary table uses a sparse representation. It must be a number in the range [0.0, 1.0]. The value 1.0 (the default), indicates that a sparse representation should be used if there are any zero-entries in the table. The value 0.0 will prevent the sparse representation from being used entirely. Sparse tables typically decrease execution time, but they use more memory. When the update approach is set to AUTOMATIC, this option impacts both the execution time and memory allocation estimates used to choose the update approach.}

\subpara{GibbsOptions for sampled factors}

Factors connected to continuous variables that do not support exact message computation, instead use a sampled approximation (see section~\ref{sec:SumProductSolver}) where the sampling is performed using the Gibbs Solver.

For all such factors in a graph, you may set any of the Gibbs solver options described in \autoref{sec:GibbsOptions} to control how the sampling will be done. The most important of these options have different default values when used with Sum-Product. This is accomplished by setting the options on the solver graph object when it is constructed. In order to override these defaults, it is necessary to set them on the solver graph, or on the factor model or solver objects.

These options and their Sum-Product default values are:

\begin{itemize}
\item \nameref{option:GibbsOptions.numSamples}: 1000 \ifjava See SampledFactor.DEFAULT\_SAMPLES\_PER\_UPDATE\fi
\item \nameref{option:GibbsOptions.burnInScans}: 10 \ifjava See SampledFactor.DEFAULT\_BURN\_IN\_SCANS\_PER\_UPDATE\fi
\item \nameref{option:GibbsOptions.scansPerSample}: 1 \ifjava See SampledFactor.DEFAULT\_SCANS\_PER\_SAMPLE\fi
\end{itemize}

\clearpage
\subsubsection{Min-Sum Solver}

Use of the MinSum solver is specified by calling:

\ifmatlab
\begin{lstlisting}
fg.Solver = 'MinSum';
\end{lstlisting}
\fi

\ifjava
\begin{lstlisting}
SFactorGraph sgraph = fg.setSolverFactory(new MinSumSolver());
\end{lstlisting}
\fi

The MinSum solver supports only discrete variables.

\para{Min-Sum Options}
\label{sec:MinSumOptions}

The following sections list the solver-specific options for the MinSum solver.

\subpara{MinSumOptions.damping}

\dimpleOption{MinSumOptions.damping}
{double}
{0.0}
{variables and factors}
{See \nameref{option:SumProductOptions.damping}}

\subpara{MinSumOptions.maxMessageSize}

\dimpleOption{MinSumOptions.maxMessagesSize}
{integer}
{integer max}
{discrete factors}
{See \nameref{option:SumProductOptions.maxMessageSize}}

\subpara{MinSumOptions.nodeSpecificDamping}

\dimpleOption{MinSumOptions.nodeSpecificDamping}
{\ifmatlab double vector\fi \ifjava OptionDoubleList\fi}
{\textit{empty}}
{variables and factors}
{See \nameref{option:SumProductOptions.nodeSpecificDamping}}

\subpara{MinSumOptions.updateApproach}

\dimpleOption{MinSumOptions.updateApproach}
{\ifmatlab string\fi \ifjava UpdateApproach enum\fi}
{AUTOMATIC}
{discrete factors}
{See \nameref{option:SumProductOptions.updateApproach}}

\subpara{MinSumOptions.automaticExecutionTimeScalingFactor}

\dimpleOption{MinSumOptions.automaticExecutionTimeScalingFactor}
{double}
{1.0}
{discrete factors}
{See \nameref{option:SumProductOptions.automaticExecutionTimeScalingFactor}}

\subpara{MinSumOptions.automaticMemoryAllocationScalingFactor}

\dimpleOption{MinSumOptions.automaticMemoryAllocationScalingFactor}
{double}
{10.0}
{discrete factors}
{See \nameref{option:SumProductOptions.automaticMemoryAllocationScalingFactor}}

\subpara{MinSumOptions.optimizedUpdateSparseThreshold}

\dimpleOption{MinSumOptions.optimizedUpdateSparseThreshold}
{double}
{1.0}
{discrete factors}
{See \nameref{option:SumProductOptions.optimizedUpdateSparseThreshold}}

\clearpage
\subsubsection{Junction Tree Solver}
\label{sec:JunctionTreeSolverAPI}

There are two distinct forms of Junction Tree solver in Dimple: the sum-product form used for computing exact marginal variable beliefs and the min-sum form used for computing MAP values.  The Junction Tree solvers support only discrete variables.

Use of one of the two forms of Junction Tree solver by calling either:

\ifmatlab
\begin{lstlisting}
fg.Solver = 'JunctionTree';
\end{lstlisting}
\fi

\ifjava
\begin{lstlisting}
JunctionTreeSolverGraph sgraph =
    fg.setSolverFactory(new JunctionTreeSolver());
\end{lstlisting}
\fi

for the sum-product version or

\ifmatlab
\begin{lstlisting}
fg.Solver = 'JunctionTreeMAP';
\end{lstlisting}
\fi

\ifjava
\begin{lstlisting}
JunctionTreeMAPSolverGraph sgraph =
    fg.setSolverFactory(new JunctionTreeMAPSolver());
\end{lstlisting}
\fi

for the min-sum version.

The Junction Tree solvers are useful for performing exact inference on loopy discrete graphical models for which the standard sum-product or min-sum algorithms will only produce approximate results. This works by transforming the model into a corresponding non-loopy model, building a proxy solver layer that connects the original model to the transformed version and doing inference on that model. If your model already is non-loopy then you can simply use sum-product or min-sum directly for exact inference. To test to see if a graph is loopy or not, use \lstname{isForest()}:

\ifmatlab
\begin{lstlisting}
useJunctionTree = ~fg.isForest();
\end{lstlisting}
\fi

\ifjava
\begin{lstlisting}
useJunctionTree = !fg.isForest();
\end{lstlisting}
\fi

One significant limitation when using this solver is that the cost of inference and amount of memory needed to store the factor tables is proportional to the size of the tables, which in turn is exponential in the number of variables represented in a table. The Junction Tree algorithm may be unable to determine an equivalent tree structure that has small enough factors either to fit in memory or to perform inference on in an acceptable amount of time. The typical failure mode in such cases is to get an \lstname{OutOfMemoryError} when attempting to run solve. The Junction Tree algorithm works best when used with smaller graphs or larger graphs that have few loops or are loopy but long and narrow.

\para{Junction Tree Options}
\label{sec:JunctionTreeOptions}

The following options are available for use with both versions of the Junction Tree solver.

Because exact inference can be done in a single iteration, the Junction Tree solver fixes the number of iterations to one and will ignore attempts to set it to another value. Because damping would result in inexact inference, the Junction Tree solver does not provide options for using damping.

\subpara{JunctionTreeOptions.useConditioning}

\dimpleOption{JunctionTreeOptions.useConditioning}
{boolean}
{false}
{graph}
{Specifies whether to use conditioning when constructing the transformation of the model. When true, then any variables in the model that have a fixed value will be disconnected from the rest of the graph in the transformed version and its value will be incorporated in the factors of the transformed model. This will produce a more efficient transformed model when there are fixed values in the original model. Using this will require that a new transformation be computed every time a fixed value changes.}

\subpara{JunctionTreeOptions.maxTransformationAttempts}

\dimpleOption{JunctionTreeOptions.maxTransformationAttempts}
{integer}
{1}
{graph}
{Specifies the maximum number of times the junction tree transformer should try to determine an optimal transformation. Each attempt uses a greedy "variable elimination" algorithm using a randomly chosen cost function and random choices to break ties, so more iterations could produce a more efficient tree transformation.}


\clearpage
\subsubsection{Gibbs Solver}
\label{sec:GibbsSolverAPI}

Use of the Gibbs solver is specified by calling:

\ifmatlab
\begin{lstlisting}
fg.Solver = 'Gibbs';
\end{lstlisting}
\fi


\ifjava
\begin{lstlisting}
GibbsSolverGraph sfg = fg.setSolverFactory(new GibbsSolver());
\end{lstlisting}
\fi

The Gibbs solver supports both discrete and continuous variables.

This solver performs Gibbs sampling on a factor graph.  It supports a variety of output information on the sampled graph, including the best joint sample (lowest energy), marginals of each variable (discrete variables only), and a full set of samples for a user-selected set of variables.  The solver supports both sequential and randomized scan, and it supports tempering with an exponential annealing schedule.

The Gibbs solver supports several user selectable generic samplers (those that don't require specific conjugacy relationships).  The following table lists the available generic samplers, and the variable types supported by each.

\begin{longtable} {l l p{8.0cm}}
Sampler & Variable Type &  Description \\
\hline
\endhead
CDFSampler & Discrete\footnote{In this table, Discrete support implies any of the discrete variable types, including Discrete and Bit.} & Samples from the full conditional distribution of the variable.  This is the default sampler for discrete variables. \\
SliceSampler & Real\footnote{In this table, Real support implies any of the continuous variable data types, including Real, Complex, and RealJoint.} & Slice sampler using the doubling procedure.  See Neal, Slice Sampling (2000).  This is the default sampler for real variables. \\
MHSampler & Discrete \& Real & Metropolis-Hastings sampler.  For discrete variables, the default proposal kernel is uniform over values other than the current value.  For real variables, the default proposal kernel is Normal with standard deviation 1 (the standard deviation is user settable).  Alternate proposal kernels are also available (see below). \\
SuwaTodoSampler & Discrete & Suwa-Todo sampler.  See Suwa, Todo, Markov Chain Monte Carlo Method without Detailed Balance (2010). \\
BlockMHSampler & Discrete \& Real & Block Metropolis-Hastings sampler.  Allows block proposals for a collection of more than one variable at a time.  In the current version of Dimple, there are no built-in general purpose block proposal kernels.  To use this sampler, a custom block proposal kernel must be written in Java, and used by specifying a block schedule entry that references this proposal kernel.  See section~\ref{sec:BlockScheduleEntries} for more information.
\end{longtable}


In cases where the factors of the graph support a conjugate distribution, the solver will automatically determine this and use the appropriate conjugate sampler.  The following table lists the supported conjugate samplers and the corresponding factors that support them.  The corresponding sampler will be used for a given variable if all of the edges connected to that variable support the same sampler\footnote{Additionally, for the conjugate sampler to be used, the domain of the variable must not be bounded to a range smaller than the natural range of the corresponding distribution.}.

\begin{longtable} {l p{7.1cm} p{2.5cm}}
Sampler & Built-in Factor & Edge \\
\hline
\endhead
BetaSampler & Beta & value \\
 & Binomial & $\rho$ \\
DirichletSampler & Dirichlet & value \\
 & Categorical & $\alpha$ \\
 & DiscreteTransition & $\alpha$ \\
GammaSampler & Gamma & value, $\beta$ \\
 & NegativeExpGamma & $\beta$ \\
 & Normal & $\tau$ \\
 & LogNormal & $\tau$ \\
 & Poisson & $\lambda$ \\
 & CategoricalUnnormalizedParameters & $\alpha$ \\
 & DiscreteTransitionUnnormalizedParameters & $\alpha$ \\
NegativeExpGammaSampler & NegativeExpGamma & value \\
 & CategoricalEnergyParameters & $\alpha$ \\ 
 & DiscreteTransitionEnergyParameters & $\alpha$ \\ 
NormalSampler & Normal & value, $\mu$ \\
 & LogNormal & $\mu$ \\
\end{longtable}

Additionally, conjugate sampling is supported across a subset of applicable deterministic functions.  In the current version of Dimple, this includes the following factors:

\begin{longtable} {l p{2cm} p{9cm}}
Factor Function & Edges & Condition \\
\hline
\endhead
Multiplexer & in, out & If any of the \emph{in} variables would support the same conjugate sampler as the \emph{out} variable, then those \emph{in} variables will use conjugate sampling. \\
\end{longtable}


The Gibbs solver automatically performs block-Gibbs updates for variables that are deterministically related.  The Gibbs solver automatically detects deterministic relationships associated with built-in deterministic factor functions (see section~\ref{sec:builtInFactors} for a list of these functions).  

\ifmatlab
For user-defined factors specified by MATLAB factor functions or factor tables, the Gibbs solver will detect if they are deterministic functions as along as the factor is marked as the directed outputs are indicated using the DirectedTo property, as described in section~\ref{sec:Factor.DirectedTo}.
\fi

\ifjava
For user-defined factors specified by factor functions or factor tables, the Gibbs solver will detect if they are deterministic functions as along as the factor is marked as the directed outputs are indicated using the DirectedTo property, as described in section~\ref{sec:Factor.DirectedTo}.
\fi
 
The Gibbs solver also automatically performs block-Gibbs updates for certain built-in factors that Gibbs sampling on individual variables would fail due to the dependencies between variables imposed by the factor.  In these cases, a custom proposal distribution ensures that proposals are consistent with the constraints of the factor.  However, the custom proposals are not assured to result in efficient mixing.  In the current version of Dimple, the following built-in factors automatically implement block-Gibbs updates:
%
\begin{itemize}
\item Multinomial
\item MultinomialUnnormalizedParameters
\item MultinomialEnergyParameters
\end{itemize}


Most Dimple methods work more-or-less as normal when using the Gibbs solver, but in some cases the interpretation is slightly different than for other solvers. For example, the .Belief method for a discrete variable returns an estimate of the belief based on averaging over the sample values.

NOTE: The setNumIterations() method is not supported by the Gibbs solver as the term ``iteration'' is ambiguous in this case. Instead, the method setNumSamples() should be used to set the length of the run. The Solver.iterate() method performs a single-variable update in the case of the Gibbs solver, rather than an entire scan of all variables.

The following sections list the solver-specific aspects of the API for the Gibbs solver.

\para{Gibbs Options}
\label{sec:GibbsOptions}

The following options affect the behavior of various aspects of the Gibbs solver:

\subpara{GibbsOptions.burnInScans}

\dimpleOption{GibbsOptions.burnInScans}
{integer}
{0}
{graph}
{Specifies the number of full updates of all of the variables in the graph to perform during the burn-in phase before generating samples.}

\subpara{GibbsOptions.discreteSampler}

\dimpleOption{GibbsOptions.discreteSampler}
{\ifmatlab string\fi \ifjava IGenericSampler class\fi}
{CDFSampler}
{variables}
{Specifies the default sampler to use for discrete variables when a conjugate sampler is not  suitable. The sampler may be configured by use of additional options defined by each sampler type. See \autoref{sec:Samplers} for more details.}

\subpara{GibbsOptions.enableAutomaticConjugateSampling}

\dimpleOption{GibbsOptions.enableAutomaticConjugateSampling}
{boolean}
{true}
{variables}
{Specifies whether to use conjugate sampling when available for a given variable. Note that if a specific sampler has been specified for a particular variable (by setting the \nameref{option:GibbsOptions.realSampler} option directly on the model or solver variable object) then a conjugate sampler will not be used regardless.}

\subpara{GibbsOptions.enableAnnealing}

\dimpleOption{GibbsOptions.enableAnnealing}
{boolean}
{false}
{graph}
{Specifies whether to use a tempering and annealing process when running Gibbs.}

\subpara{GibbsOptions.initialTemperature}

\dimpleOption{GibbsOptions.initialTemperature}
{double}
{1.0}
{graph}
{Specifies the initial temperature to use when annealing is enabled (as specified by the enableAnnealing option).}

\subpara{GibbsOptions.numSamples}

\dimpleOption{GibbsOptions.numSamples}
{integer}
{1}
{graph}
{Specifies the number of samples to be generated when solving the graph post burn-in. This value times the value of \nameref{option:GibbsOptions.numRandomRestarts} plus one determines the number of samples that will be produced in a call to solveOneStep.}

\subpara{GibbsOptions.numRandomRestarts}

\dimpleOption{GibbsOptions.numRandomRestarts}
{integer}
{0}
{graph}
{Specifies the number of random restarts (zero by default, which means run once and don't restart). For a value greater than zero, the after running the specified number of samples, the solver is restarted with the variable values randomized, and re-run (including burn-in).  The sample values (the best sample value, or all samples, if requested) are extracted across all runs.
}

\subpara{GibbsOptions.realSampler}

\dimpleOption{GibbsOptions.realSampler}
{\ifmatlab string\fi \ifjava IGenericSampler class\fi}
{SliceSampler}
{variables}
{Specifies the default sampler to use for real-valued variables (including Complex and RealJoint) when a conjugate sampler is not suitable. The sampler may be configured by use of additional options defined by each sampler type. See \autoref{sec:Samplers} for more details.}

\subpara{GibbsOptions.scansPerSample}

\dimpleOption{GibbsOptions.scansPerSample}
{integer}
{0}
{graph}
{Specifies sampling rate in terms of the number of updates of all of the variables in the graph to perform for each sample.}


\subpara{GibbsOptions.saveAllSamples}

\dimpleOption{GibbsOptions.saveAllSamples}
{boolean}
{false}
{graph}
{Specifies whether to save all sample values for variables when running Gibbs. Note that this is practical only if the number of variables in the graph times the number of samples per variable is reasonably sized.}

\subpara{GibbsOptions.saveAllScores}

\dimpleOption{GibbsOptions.saveAllScores}
{boolean}
{false}
{graph}
{Specifies whether to save scores for all generated samples in Gibbs. If true, then for each sample the total energy/log-likelihood aka \emph{score} of the graph will be saved. The saved scores can later be retrieved by the getAllScores() method described below.}

\subpara{GibbsOptions.annealingHalfLife}

\dimpleOption{GibbsOptions.annealingHalfLife}
{double}
{1.0}
{graph}
{Specifies the rate at which the temperature will be lowered during the tempering and annealing process. This rate is specified in terms of the number of iterations required for the temperature to be lowered by half. This value is only used if annealing has been enabled as specified by the enableAnnealing option.}

\para{Graph Methods}

The following methods are available on a graph set to use the Gibbs solver:

Variable initialization (both on the first run and subsequent restarts) is randomized whenever possible.  For a discrete variable, the value is sampled from the Input (uniform if an input is not specified).  For a real variable, if an Input is specified and the Input supports one of the conjugate samplers listed above, that sampler is used to initialize the variable.  If bounds are also specified for the variable domain, the values is truncated to fall within the bounds.  If only bounds are specified (which are finite above and below), then the value is uniformly sampled from within the bounds.  If no finite bounds are specified and there is no input, the variable is initialized to zero (or the value specified by setInitialSampleValue) on the initial run, and left at the final value of the previous run on restart.

Enable or disable the use of tempering, or determine if tempering is in use.

\ifmatlab
\begin{lstlisting}
graph.Solver.setTemperature(T);
graph.Solver.getTemperature();
\end{lstlisting}
\fi

\ifjava
\begin{lstlisting}
sfg.setTemperature(T);
sfg.getTemperature();
\end{lstlisting}
\fi

Set/get the current temperature. Setting the current temperature overrides the current annealing temperature.

\ifmatlab
\begin{lstlisting}
graph.Solver.getAllScores();
\end{lstlisting}
\fi

\ifjava
\begin{lstlisting}
sfg.getAllScores();
\end{lstlisting}
\fi

Returns an array including the score value for each sample. This method only returns a non-empty value if the \nameref{option:GibbsOptions.saveAllScores} option was set to true on the graph before generating samples.

\ifmatlab
\begin{lstlisting}
graph.Solver.getTotalPotential();
\end{lstlisting}
\fi

\ifjava
\begin{lstlisting}
sfg.getTotalPotential();
\end{lstlisting}
\fi

After running the solver, returns the total potential (score) over all factors of the graph (including input priors on variables) given the most recent sample values.

\ifmatlab
\begin{lstlisting}
graph.Solver.sample(numSamples)
\end{lstlisting}
\fi

\ifjava
\begin{lstlisting}
sfg.sample(numSamples)
\end{lstlisting}
\fi

This method runs a specified number of samples without re-initializing, burn-in, or random-restarts (this is distinct from iterate(), which runs a specified number of single-variable updates).  Before running this method for the first time, the graph must be initialized using the initialize() method.

\ifmatlab
\begin{lstlisting}
graph.Solver.burnIn()
\end{lstlisting}
\fi

\ifjava
\begin{lstlisting}
sfg.burnIn()
\end{lstlisting}
\fi

Run the burn-in samples independently of using solve (which automatically runs the burn-in samples).  This may be run before using sample() or iterate().

\ifmatlab
\begin{lstlisting}
graph.Solver.getRejectionRate()
\end{lstlisting}
\fi

\ifjava
\begin{lstlisting}
sfg.getRejectionRate()
\end{lstlisting}
\fi

Get the overall rejection rate over the entire graph.  The rejection rate is the ratio of the number of MCMC proposals that were rejected to the total number of sampler updates.  Rejections only occur in MCMC samplers, such as the MHSampler.  In other samplers, such as conjugate samplers or the CDFSampler, rejection doesn't occur and the rate for variables that use these samplers is zero (in these cases, sampling the same value twice in a row is not considered rejection).  When getting the rejection rate for the entire graph, both the number of rejections and number of updates is counted for all variables, as well as all blocks of variables over which a block sampler is used.  These counts are accumulated from the time the graph is initialized, which automatically occurs when running solve().  This includes both burn-in and subsequent sampling.  These counts can also be reset explicitly using the resetRejectionRateStats method (see below), which allows these values to be determined, for example, during a specific set of samples.

\ifmatlab
\begin{lstlisting}
graph.Solver.resetRejectionRateStats()
\end{lstlisting}
\fi

\ifjava
\begin{lstlisting}
sfg.resetRejectionRateStats()
\end{lstlisting}
\fi

Explicitly reset the rejection-rate statistics.  These are automatically reset when the graph is initialized, which automatically occurs when running solve(), but may be reset manually at other times using this method.


\para{Variable Methods}

\ifmatlab
\begin{lstlisting}
variable.Solver.getCurrentSample();
\end{lstlisting}
\fi

\ifjava
\begin{lstlisting}
Object d = ((GibbsDiscrete)variable.getSolver()).getCurrentSample();
double r = ((GibbsReal)variable.getSolver()).getCurrentSample();
double[] rj = ((GibbsRealJoint)variable.getSolver()).getCurrentSample();
Value val= ((ISolverVariableGibss)variable.getSolver()).getCurrentSampleValue();
\end{lstlisting}
\fi

Returns the current sample value for a variable.

\ifmatlab
\begin{lstlisting}
variable.Solver.getAllSamples();
\end{lstlisting}
\fi

\ifjava
\begin{lstlisting}
Object[] ds = ((GibbsDiscrete)variable.getSolver()).getAllSamples();
double[] rs = ((GibbsReal)variable.getSolver()).getAllSamples();
double[][] rjs = ((GibbsRealJoint)variable.getSolver()).getAllSamples();
\end{lstlisting}
\fi

Returns an array including all sample values seen so far for a variable. Over multiple variables, samples with the same index correspond to the same joint sample value. This method only returns a non-empty value if the \nameref{GibbsOptions.saveAllSamples} options was enabled for the variable when sampling was performed.

\ifmatlab
\begin{lstlisting}
variable.Solver.getBestSample();
\end{lstlisting}
\fi

\ifjava
\begin{lstlisting}
Object d = ((GibbsDiscrete)variable.getSolver()).getBestSample();
double r = ((GibbsReal)variable.getSolver()).getBestSample();
double[] rj = ((GibbsRealJoint)variable.getSolver()).getBestSample();
\end{lstlisting}
\fi


Returns the value of the best sample value seen so far, where best is defined as the sample with the minimum total potential over the graph (sum of -log of the factor values and input priors).  When getting the best sample from multiple variables, they all correspond to the same sample in time, thus should be a valid sample from the joint distribution.

\ifmatlab
\begin{lstlisting}
variable.Solver.setInitialSampleValue(initialSampleValue)
variable.Solver.getInitialSampleValue()
\end{lstlisting}
\fi

\ifjava
\begin{lstlisting}
variable.getSolver().setInitialSampleValue(initialSampleValue)
variable.getSolver().getInitialSampleValue()
\end{lstlisting}
\fi

Set/get the initial sample value to be used as the starting value for this variable.  This value is used only on the first run (not subsequent restarts).  Setting this value overrides any randomization of the starting value on the first run.


\ifmatlab
\begin{lstlisting}
variable.Solver.setSampler(samplerName);
variable.Solver.getSampler();
variable.Solver.getSamplerName();
\end{lstlisting}
\fi

\ifjava
\begin{lstlisting}
variable.getSolver().setSampler(samplerName);
variable.getSolver().getSampler();
variable.getSolver().getSamplerName();
\end{lstlisting}
\fi

Set/get the sampler to be used for this variable.  Setting the sampler for a given variable overrides the default sampler for the given variable type, and also overrides any conjugate sampler that might otherwise be used.  Using this method the sampler may be set only to one of the generic samplers appropriate for the given variable type.

The getSampler method returns the sampler object, while the getSamplerName method returns a string indicating the name of the sampler being used for this variable.  Automatic assignment of a conjugate sampler is done at graph initialization time, so in order to determine what sampler will actually be used, these methods must be called either after a call to the graph initialize method, or after running the solver.

\ifmatlab
\begin{lstlisting}
variable.Solver.getRejectionRate()
\end{lstlisting}
\fi

\ifjava
\begin{lstlisting}
variable.getSolver().getRejectionRate()
\end{lstlisting}
\fi

Get the rejection rate for the sampler used for a specific variable.  The rejection rate is the ratio of the number of MCMC proposals that were rejected to the total number of sampler updates.  Rejections only occur in MCMC samplers, such as the MHSampler.  In other samplers, such as conjugate samplers or the CDFSampler, rejection doesn't occur and the rate for variables that use these samplers is zero (in these cases, sampling the same value twice in a row is not considered rejection).  These counts are accumulated from the time the graph is initialized, which automatically occurs when running solve().  This includes both burn-in and subsequent sampling.  These counts can also be reset explicitly using the resetRejectionRateStats method (see below), which allows these values to be determined, for example, during a specific set of samples.


\ifmatlab
\begin{lstlisting}
variable.Solver.getNumScoresPerUpdate()
\end{lstlisting}
\fi

\ifjava
\begin{lstlisting}
variable.getSolver().getNumScoresPerUpdate()
\end{lstlisting}
\fi

Returns the average number of score computations performed per update when sampling from this variable.  Use of an MCMC sampler requires computation of the score (energy) for specific settings of the variables.  For some samplers, such as the slice sampler, the number of times the score is computed varies, and depends on the particular values and the form of the distribution.  The returned value indicates the average number of times the score has been computed.  If a non-MCMC-based sampler is used, the returned value will be zero.  The count is accumulated from the time the graph is initialized, which automatically occurs when running solve().  This includes both burn-in and subsequent sampling.  The count can also be reset explicitly using the resetRejectionRateStats method (see below), which allows this value to be determined, for example, during a specific set of samples.

\ifmatlab
\begin{lstlisting}
variable.Solver.resetRejectionRateStats()
\end{lstlisting}
\fi

\ifjava
\begin{lstlisting}
variable.getSolver().resetRejectionRateStats()
\end{lstlisting}
\fi

Explicitly reset the rejection-rate statistics for a specific variable.  These are automatically reset when the graph is initialized, which automatically occurs when running solve(), but may be reset manually at other times using this method.



\para{Discrete-Variable-Specific Methods}

The following methods apply only to Discrete, Bit, and FiniteField variables when using the Gibbs solver.

\ifmatlab
\begin{lstlisting}
variable.Solver.getSampleIndex;
\end{lstlisting}
\fi

\ifjava
\begin{lstlisting}
variable.getSolver().getSampleIndex();
\end{lstlisting}
\fi

Returns the index of the current sample for a variable, where the index refers to the index into the domain of the variable.

\ifmatlab
\begin{lstlisting}
variable.Solver.getAllSampleIndices;
\end{lstlisting}
\fi

\ifjava
\begin{lstlisting}
variable.getSolver().getAllSampleIndices();
\end{lstlisting}
\fi

Returns an array including the indices of all samples seen so far for a variable.

\ifmatlab
\begin{lstlisting}
variable.Solver.getBestSampleIndex;
\end{lstlisting}
\fi

\ifjava
\begin{lstlisting}
variable.getSolver().getBestSampleIndex();
\end{lstlisting}
\fi

Returns the index of the best sample seen so far.


\ifmatlab
\begin{lstlisting}
variable.Solver.setInitialSampleIndex(initialSampleIndex)
variable.Solver.getInitialSampleIndex()
\end{lstlisting}
\fi

\ifjava
\begin{lstlisting}
variable.getSolver().setInitialSampleIndex(initialSampleIndex)
variable.getSolver().getInitialSampleIndex()
\end{lstlisting}
\fi

Set/get the initial sample index associated with the starting value for this variable.  The value associated with this index is used only on the first run (not subsequent restarts).  Setting this index overrides any randomization of the starting value on the first run.

\para{Real-Variable-Specific Methods}

The following methods apply only to Real variables when using the Gibbs solver.

\ifmatlab
\begin{lstlisting}
variable.Solver.getSampleMean;
\end{lstlisting}
\fi

\ifjava
\begin{lstlisting}
variable.getSolver().getSampleMean();
\end{lstlisting}
\fi

Returns the mean value of all samples that have been collected.  This is and estimate of the mean of the belief for the corresponding variable.

\ifmatlab
\begin{lstlisting}
variable.Solver.getSampleVariance;
\end{lstlisting}
\fi

\ifjava
\begin{lstlisting}
variable.getSolver().getSampleVariance();
\end{lstlisting}
\fi

Returns the variance of all samples that have been collected.  This is and estimate of the variance of the belief for the corresponding variable.

\para{RealJoint-Variable-Specific Methods}

The following methods apply only to RealJoint and Complex variables when using the Gibbs solver.

\ifmatlab
\begin{lstlisting}
variable.Solver.getSampleMean;
\end{lstlisting}
\fi

\ifjava
\begin{lstlisting}
variable.getSolver().getSampleMean();
\end{lstlisting}
\fi

Returns the mean vector of all samples that have been collected.  This is and estimate of the mean of the belief for the corresponding variable.  This method is only available if, prior to performing inference, the option GibbsOptions.computeRealJointBeliefMoments is set to true.

\ifmatlab
\begin{lstlisting}
variable.Solver.getSampleCovariance;
\end{lstlisting}
\fi

\ifjava
\begin{lstlisting}
variable.getSolver().getSampleCovariance();
\end{lstlisting}
\fi

Returns the covariance matrix computed over all samples that have been collected.  This is and estimate of the covariance of the belief for the corresponding variable.  This method is only available if, prior to performing inference, the option GibbsOptions.computeRealJointBeliefMoments is set to true.

\para{Factor Methods}

\ifmatlab
\begin{lstlisting}
factor.Solver.getPotential();
\end{lstlisting}
\fi

\ifjava
\begin{lstlisting}
factor.getSolver().getPotential();
\end{lstlisting}
\fi

Returns the potential value of a factor given the current values of its connected variables.

\ifmatlab
\begin{lstlisting}
factor.Solver.getPotential(values);
\end{lstlisting}
\fi

\ifjava
\begin{lstlisting}
factor.getSolver().getPotential(values);
\end{lstlisting}
\fi

Get the potential value of a factor given the variable values specified by the argument vector. The argument must be a vector with length equal to the number of connected variables. For a table-factor (connected exclusively to discrete variables), each value corresponds the index into the domain list for that variable (not the value of the variable itself). For a real-factor (connected to one or more real variables), each value corresponds to the value of the variable.

\para{Schedulers and Schedules}

The built-in schedulers designed for belief propagation are not appropriate for the Gibbs solver.  Instead, there are two built-in schedulers specifically for the Gibbs solver:

\begin{itemize}
\item GibbsSequentialScanScheduler
\item GibbsRandomScanScheduler
\end{itemize}

The GibbsSequentialScanScheduler chooses the next variable for updating in a fixed order. It updates all variables in the graph, completing an entire scan, before repeating the same fixed order. (In Gibbs literature this seems to be known as a sequential-scan, systematic-scan, or fixed-scan schedule.)

The GibbsRandomScanScheduler randomly selects a variable for each update (with replacement).

The default scheduler when using the Gibbs solver is the GibbsSequentialScanScheduler, which is used if no scheduler is explicitly specified.

The user may specify a custom schedule when using the Gibbs solver.  In this case, the schedule should include only Variable node updates (not specific edges), and no Factor updates (any Factor updates specified will be ignored).

To explicitly specify a scheduler, use the Scheduler or Schedule property of the FactorGraph (see sections~\ref{sec:FactorGraph.Scheduler} and~\ref{sec:FactorGraph.Schedule}).

\subpara{Block Schedule Entries}
\label{sec:BlockScheduleEntries}

The Gibbs solver allows a schedule to optionally include block entries that allow a group of variables to be updated at once.  A block schedule entry can either be included in a custom schedule or added to the schedule produced by one of the Gibbs-specific built-in schedulers.  A block schedule entry includes two pieces of information:
%
\begin{itemize}
\item A reference to a block sampler, which is used to perform the update
\item A list of variables to be included in this block
\end{itemize}

In the current version of Dimple, the only built-in block sampler is the BlockMHSampler, which implements block Metropolis-Hastings sampling for the variables included in the block.  The BlockMHSampler requires a block proposal kernel to be specified.  In the current version of Dimple, there are no built-in general purpose block proposal kernels.  To use this sampler, a custom block proposal kernel must be written \ifmatlab in Java \fi (see section~\ref{sec:CreatingACustomProposalKernel}).

\ifmatlab
To specify a block schedule entry in a custom schedule, the schedule entry consists of a cell-array in which the first element of the cell is a reference to a Java sampler object, and the subsequent entries are the variables to be included in the block.  For example:

\begin{lstlisting}
import com.analog.lyric.dimple.solvers.gibbs.samplers.block.BlockMHSampler;
...
fg.Schedule = {{BlockMHSampler(MyProposalKernel), a, b, c}, d, e};
\end{lstlisting}
\fi

\ifjava
To add a block schedule entry to a custom schedule, the the argument to the \texttt{add} method is a reference to a sampler object, and the subsequent entries are the variables to be included in the block.  For example:

\begin{lstlisting}
FixedSchedule s = new FixedSchedule();
s.add(new BlockScheduleEntry(new BlockMHSampler(new MyProposalKernel()), a, b, c));
s.add(new NodeScheduleEntry(d));
s.add(new NodeScheduleEntry(e));
fg.setSchedule(s);
\end{lstlisting}
\fi

In the above example, we create a block schedule entry that updates variables a, b, and c together, with separate schedule entries for variables d and e.  The constructor for the BlockMHSampler requires a proposal kernel.  In the above example, ``MyProposalKernel'' is a user-provided custom proposal class \ifmatlab written in Java\fi.  \ifmatlab (Note that the ``import'' line in the above example is simply to avoid having to write the fully qualified name each time the BlockMHSampler is used.) \fi

Block schedule entries can also be used with either of the Gibbs-specific built-in schedulers described above.  When a block entry is added in this way, for each of the variables included in a block entry, the individual variable entries that would have been present in the schedule are removed.  That is, those variables are only included in the corresponding block entry (or entries) and are not also updated independently.  In case of the GibbsRandomScanScheduler, each update selects an entry randomly from among all blocks plus all variables that are not in a block.

\ifmatlab
A block schedule entry can be added when using a built-in Gibbs-specific scheduler using:
\begin{lstlisting}
fg.Scheduler.addBlockScheduleEntry(blockSampler, listOfVariables);
\end{lstlisting}

The following example shows adding a block schedule entry that includes two elements of the variable x, and variable y.

\begin{lstlisting}
fg.Scheduler.addBlockScheduleEntry(BlockMHSampler(MyKernel), x(2:3), y);
\end{lstlisting}


Multiple block schedule entries can be added at once using:
\begin{lstlisting}
fg.Scheduler.addBlockScheduleEntries({blockSampler1, listOfVariables1}, {blockSampler1, listOfVariables1}, ...);
\end{lstlisting}

The following example shows adding a block schedule entry that includes two elements of the variable x, and variable y, and a second block that includes variables a and b.

\begin{lstlisting}
fg.Scheduler.addBlockScheduleEntries({BlockMHSampler(MyKernel1), x(2:3), y}, {BlockMHSampler(MyKernel2), a, b});
\end{lstlisting}

\fi

\ifjava
A block schedule entry can be added when using a built-in Gibbs-specific scheduler using:

\begin{lstlisting}
((IGibbsScheduler)fg.getScheduler()).addBlockScheduleEntry(new BlockScheduleEntry(blockSampler, listOfVariables));
\end{lstlisting}

The following example shows adding a block schedule entry that includes the variables a, b, and c.

\begin{lstlisting}
((IGibbsScheduler)fg.getScheduler()).addBlockScheduleEntry(new BlockScheduleEntry(new BlockMHSampler(new MyProposalKernel()), a, b, c));
\end{lstlisting}

Note that to use the addBlockScheduleEntry method, the scheduler must first have been explicitly set to one of the Gibbs-specific schedulers.
\fi

To implement a custom block proposal kernel, a new Java class must be created that implements the IBlockProposalKernel interface.  See section~\ref{sec:CreatingACustomProposalKernel} for more detail.

\clearpage
\subsubsection{Particle BP Solver}

Use of the particle BP solver is specified by calling:

\ifmatlab
\begin{lstlisting}
fg.Solver = 'ParticleBP';
\end{lstlisting}
\fi
\ifjava
\begin{lstlisting}
ParticleBPSolverGraph sfg = fg.setSolverFactory(new ParticleBPSolver());
\end{lstlisting}
\fi

The following lists the solver-specific options for the ParticleBP solver.

\para{Particle BP Options}

\subpara{ParticleBPOptions.numParticles}

\dimpleOption{ParticleBPOptions.numParticles}
{integer}
{1}
{real variables}
{Specifies the number of particles used to represent the variable. This option takes affect when the solver variable is constructed and when it is initialized.}

\subpara{ParticleBPOptions.enableAnnealing}

\dimpleOption{ParticleBPOptions.enableAnnealing}
{boolean}
{false}
{graph}
{Determines whether to use a tempering and annealing process during inference.}

\subpara{ParticleBPOptions.initialTemperature}

\dimpleOption{ParticleBPOptions.initialTemperature}
{double}
{1.0}
{graph}
{Specifies the initial temperature to use when annealing is enabled (as specified by the enableAnnealing option).}

\subpara{ParticleBPOptions.annealingHalfLife}

\dimpleOption{ParticleBPOptions.annealingHalfLife}
{double}
{1.0}
{graph}
{Specifies the rate at which the temperature will be lowered during the tempering and annealing process. This rate is specified in terms of the number of iterations required for the temperature to be lowered by half. This value is only used if annealing has been enabled as specified by the enableAnnealing option.}

\subpara{ParticleBPOptions.resamplingUpdatesPerParticle}

\dimpleOption{ParticleBPOptions.resamplingUpdatesPerParticle}
{integer}
{1}
{real variables}
{For variables on which it is set, specifies the number of updates per particle to perform each time the particle is resampled.}

\subpara{ParticleBPOptions.iterationsBetweenResampling}

\dimpleOption{ParticleBPOptions.iterationsBetweenResampling}
{integer}
{1}
{graph}
{Specifies the number of iterations between re-sampling all of the variables in the graph. Default is 1, meaning resample between every iteration.}

\subpara{ParticleBPOptions.initialParticleRange}

\dimpleOption{ParticleBPOptions.initialParticleRange}
{\ifmatlab 1x2 vector\fi \ifjava two-element OptionDoubleList\fi}
{[-infinity infinity]}
{variables}
{Set the range over which the initial particle values will be defined. The initial particle values are uniformly spaced between the min and max values specified. If the range is specified using this method, it overrides any other initial value. Otherwise, if a finite domain has been specified, the initial particle values are uniformly spaced between the lower and upper bound of the domain. Otherwise, all particles are initially set to zero.}

\ifmatlab
\begin{lstlisting}
% Set particle range to the unit interval for all variables in the graph:
graph.setOption('ParticleBPOptions.initialParticleRange', [0.0, 1.0]);
\end{lstlisting}
\fi
\ifjava
\begin{lstlisting}
// Set particle range to the unit interval for all variables in the graph:
ParticleBPOptions.initialParticleRange.set(graph, 0.0, 1.0);
\end{lstlisting}
\fi

\subpara{ParticleBPOptions.proposalKernel}

\dimpleOption{ParticleBPOptions.proposalKernel}
{\ifmatlab string\fi \ifjava IProposalKernel class\fi}
{NormalProposalKernel}
{real variables}
{Specifies the type of proposal kernel to use for the specified variables. The selected proposal kernel may have additional options that can be used to configure its behavior. These can also be set on the variables.}

\para{Graph Methods}

The following solver-specific methods are available on the solver graph. \ifjava (It is assumed that sfg is a variable of type ParticleBPSolverGraph obtained when the solver was set or by casting the result of the getSolver() method.)\fi

\ifmatlab
\begin{lstlisting}
graph.Solver.setTemperature(newTemperature);
temp = graph.Solver.getTemperature();
\end{lstlisting}
\fi

\ifjava
\begin{lstlisting}
sfg.setTemperature(newTemperature);
double temp = sfg.getTemperature();
\end{lstlisting}
\fi

Set/get the current temperature. Setting the current temperature overrides the current annealing temperature. This should rarely be necessary.

\para{Variable Methods}

The Particle BP solver supports both discrete and real variables. For discrete variables, the solver uses sum-product BP as normal, and all of the corresponding methods for the sum-product solver may be used for discrete variables. For real variables, several solver-specific methods are defined, as follows.

\para{Real-Variable-Specific Methods}
\label{sec:ParticleBPRealVariableSpecificMethods}

\ifmatlab
\begin{lstlisting}
variable.Solver.setProposalStandardDeviation(stdDev);
variable.Solver.getProposalStandardDeviation();
\end{lstlisting}
\fi

\ifjava
\begin{lstlisting}
variable.getSolver().setProposalStandardDeviation(stdDev);
variable.getSolver().getProposalStandardDeviation();
\end{lstlisting}
\fi


Set/get the standard deviation for a Gaussian proposal distribution (the default is 1).

\ifmatlab
\begin{lstlisting}
variable.Solver.getParticleValues();
\end{lstlisting}
\fi

\ifjava
\begin{lstlisting}
((ParticleBPReal)variable.getSolver()).getParticleValues();
\end{lstlisting}
\fi

Returns the current set of particle values associated with the variable.

\ifmatlab
\begin{lstlisting}
variable.Solver.getBelief(valueSet);
\end{lstlisting}
\fi

\ifjava
\begin{lstlisting}
((ParticleBPReal))variable.getSolver()).getBelief(valueSet);
\end{lstlisting}
\fi

Given a set of values in the domain of the variable, returns the belief evaluated at these points. The result is normalized relative to the set of points requested so that the sum over the set of returned beliefs is 1.

NOTE: the generic variable method Belief (or getBeliefs() with no arguments) operates similarly to the discrete-variable case, but the belief values returned are those at the current set of particle values. Note that this representation does not represent a set of weighted particles. That is, the particle positions are distributed approximately by the belief and the belief values represent the belief. It remains to be see if this should be the representation of belief that is used, or if an alternative representation would be better. The alternative solver-specific getBelief(valueSet) method allows getting the beliefs on a user-specified set of values, which may be uniform, and would not have this unusual interpretation.


\clearpage
\subsubsection{LP Solver}
\label{sec:LPSolver}

Use of the linear programming (LP) solver is specified by calling:

\ifmatlab
\begin{lstlisting}
fg.Solver = 'LP';
\end{lstlisting}
\fi
\ifjava
\begin{lstlisting}
fg.setSolverFactory(new LPSolver());
\end{lstlisting}
\fi

The LP solver transforms a factor graph MAP estimation problem into an equivalent linear program, which is solved using a linear programming software package. The solver can either be a linear programming solver (in which case the MAP is estimated using an LP relaxation, with no guarantees of correctness), or by an integer linear programming (ILP) solver, in which case the solution is guaranteed to be the MAP. Because this solver release on an external package, you will need to install and configure the specified package before using this solver.

The LP solver supports only discrete variables.

In the current version of Dimple (version 0.7), there is no support for rolled-up graphs when using the LP solver.

The following options are applicable to the LP solver:

\para{LP Options}

\subpara{LPOptions.LPSolver}

\dimpleOption{LPOptions.LPSolver}
{string}
{`'}
{graph}
{Selects which external LP solver will be used to solve the linear program. Valid values include `matlab', `CPLEX', `GLPK', `Gurobi', `LpSolve', `MinSate', `Mosek', and `SAT4J'. The default value is synonomous with specifying `matlab' and will delegate the solver specified by the MatlabLPOption that will be run from the MATLAB frontend. This will obviously only work when running Dimple from MATLAB. None of these solvers are included with Dimple and must be installed and configured separately. The interface for the non-MATLAB based solvers is provided by the third-party Java ILP package. See \href{http://javailp.sourceforge.net}{javailp.sourceforge.net} for more information about configuring various solvers.}

\subpara{LPSolver.MatlabLPSolver}

\dimpleOption{LPSolver.MatlabLPSolver}
{string}
{`'}
{graph}
{Selects which LP solver will be run from MATLAB to solve the linear program. This option is only relevant if the LPSolver option has been set to `matlab' and Dimple is being run from MATLAB. The choices for the string solvername are `matlab', `glpk', `glpkIP', `gurobi', and `gurobiIP'. The `matlab', `glpk', and `gurobi' solvers are linear programming solvers, while `glpkIP' and `gurobiIP' are ILP solvers. The default value is synonomous with `matlab'.

Using the matlab LP solver requires the the MATLAB Optimization Toolbox. Using `glpk' or `glpkIP' requires glpkmex to be in the matlab path, and `gurobi' and `gurobiIP' require the gurobi matlab interface to be in the matlab path; in either case the appropriate packages will need to be obtained and installed.
}

\clearpage
\subsubsection{Proposal Kernels}
\label{sec:ProposalKernels}

The following proposal kernels are provided by Dimple for use in samplers. Additional kernels may be added by creating new proposal kernel Java classes that implement the appropriate interfaces.

\para{NormalProposalKernel}
\label{sec:NormalProposalKernel}

The following options may be used to configure this kernel:

\subpara{NormalProposalKernel.standardDeviation}

\dimpleOption{NormalProposalKernel.standardDeviation}
{double}
{1.0}
{variables}
{Specifies the standard deviation to use on NormalProposalKernel instances attached to the variables that are affected by the option setting. The value must be non-negative.}

\para{CircularNormalProposalKernel}
\label{sec:CircularNormalProposalKernel}

The CircularNormalProposalKernel makes proposals from a Normal distribution on a circularly wrapping range of the real line.  For example, setting the bounds of the range to $-\pi$ and $\pi$ would create proposals representing angles on a circle.

Since this is a subclass of NormalProposalKernel, the standardDeviation option defined for that class will also affect this one. The following additional options may also be used:

\subpara{CircularNormalProposalKernel.lowerBound}

\dimpleOption{CircularNormalProposalKernel.lowerBound}
{double}
{$-\pi$}
{variables}
{Specifies lower bound to use on CircularNormalProposalKernel instances attached to the variables that are affected by the option setting.}

\subpara{CircularNormalProposalKernel.upperBound}

\dimpleOption{CircularNormalProposalKernel.upperBound}
{double}
{$+\pi$}
{variables}
{Specifies upper bound to use on CircularNormalProposalKernel instances attached to the variables that are affected by the option setting.}

\para{UniformDiscreteProposalKernel}
\label{sec:UniformDiscreteProposalKernel}

This kernel does not have any configurable options.

\clearpage
\subsubsection{Samplers}
\label{sec:Samplers}

Samplers are used by sampling solvers (e.g. Gibbs) to generate samples for variables either singly or in blocks.

The following non-conjugate single variable samplers are currently available:

\para{CDFSampler}
\label{sec:CDFSampler}

The CDFSampler can be used with discrete variable types. It samples from the full conditional distribution of the variable. It is the default sampler used for discrete variables in the Gibbs solver.

It does not support any configuration options.

\para{MHSampler}
\label{sec:MHSampler}

The MHSampler may be used for discrete or real variables. It implements the Metropolis-Hastings sampling algorithm. The sampler may be configured to use different proposal kernels for discrete and real variables as described below.

The following options may be used to configure the MHSampler for a given variable:

\subpara{MHSampler.discreteProposalKernel}

\dimpleOption{MHSampler.discreteProposalKernel}
{\ifmatlab string\fi \ifjava IProposalKernel class\fi}
{UniformDiscreteProposalKernel}
{variables}
{Specifies the proposal kernel to use when using the MHSampler on discrete variables for which this option setting is visible. Depending on the kernel selected, it may also be configured by additional option settings. See \autoref{sec:ProposalKernels} for more details.}

\subpara{MHSampler.realProposalKernel}

\dimpleOption{MHSampler.realProposalKernel}
{\ifmatlab string\fi \ifjava IProposalKernel class\fi}
{NormalProposalKernel}
{variables}
{Specifies the proposal kernel to use when using the MHSampler on real variables for which this option setting is visible. Depending on the kernel selected, it may also be configured by additional option settings. See \autoref{sec:ProposalKernels} for more details.}

\para{SliceSampler}
\label{sec:SliceSampler}

The slice sampler may be used with real variables. It implements the algorithm described in Neal's paper ``Slice Sampling" 2010. It is the default sampler used for real variables in the Gibbs solver.

The following options may be used to configure the SliceSampler for a given variable:

\subpara{SliceSampler.initialSliceWidth}

\dimpleOption{SliceSampler.initialSliceWidth}
{double}
{1.0}
{variables}
{The size of the initial slice to use when sampling for SliceSampler instances used by variables that are affected by the option setting. For variables with a natural range that is much smaller or much larger than the default value of one, it may be beneficial to modify this value.}

\subpara{SliceSampler.maximumDoublings}

\dimpleOption{SliceSampler.maximumDoublings}
{integer}
{10}
{variables}
{}

\para{SuwaTodoSampler}
\label{sec:SuwoTodoSampler}

The Suwo-Todo sampler can be used for discrete variables. It implements the algorithm described in Suwa and Todo's paper ``Markov Chain Monte Carlo Method without Detailed Balance" (2010).

This sampler does not support any configuration options.
