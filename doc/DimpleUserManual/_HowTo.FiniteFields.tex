\subsection{Using Finite Field Variables}
\label{sec:finiteFields}


\subsubsection{Overview}

Dimple supports a special variable type called a FiniteFieldVariable and a few custom factors for these variables. They represent finite fields with $N=2^{n}$ elements. These fields find frequent use in error correcting codes. Because Dimple can describe any discrete distribution, it is possible to handle finite fields simply by describing their factor tables. However, the native FiniteFieldVariable type is much more efficient. In particular, variable addition and multiplication, which naively require $\mathcal{O}(N^{3})$ operations, are calculated in only $\mathcal{O}(N\log N)$ operations.

\ifmatlab

\subsubsection{Finite Fields Without Optimizations}

As we mentioned previously, a user can construct (non-optimized) finite fields from scratch.

First we create a domain for our variables using the MATLAB gf function (for Galois Field).

\begin{lstlisting}
m = 3; 
numElements = 2^m;
domain = 0:numElements-1;

tmp = gf(domain,m); 
real_domain = cell(length(tmp),1);
for i = 1:length(tmp)
  real_domain{i} = tmp(i); 
end
\end{lstlisting}

Next we create a bunch of variables with that domain.

\begin{lstlisting}
x_slow = Discrete(real_domain);
y_slow = Discrete(real_domain); 
z_slow = Discrete(real_domain);
\end{lstlisting}

Now we create our graph and add the addition constraint.

\begin{lstlisting}
fg_slow = FactorGraph();

addDelta = @(x,y,z) x+y==z;
fg_slow.addFactor(addDelta,x_slow,y_slow,z_slow);
\end{lstlisting}

This code runs in $\mathcal{O}(N^3) $ time since it tries all combinations of x,y, and z.
Next we set some inputs.

\begin{lstlisting}
x_input = rand(size(x_slow.Domain.Elements));
y_input = rand(size(y_slow.Domain.Elements));

x_slow.Input = x_input;
y_slow.Input = y_input;
\end{lstlisting}

Finally we set number of iterations, solve, and look at beliefs.

\begin{lstlisting}
fg_slow.NumIterations = 1;

fg_slow.solve();

z_slow.Belief
\end{lstlisting}

The solver runs in $\mathcal{O}(N^2)$ time since z is determined by x and y, x is determined by z, and y, and y is determined by x and z.

\fi

\subsubsection{Optimized Finite Field Operations}

Rather than building finite field elements from scratch, a user can use a built-in variable type and associated set of function nodes. These native variables are much faster, both for programming and algorithmic reasons. All of these operations are supported with the SumProduct solver.

\para{FiniteFieldVariables}

Dimple supports a FiniteFieldVariable variable type, which takes a primitive polynomial (to be discussed later) and dimensions of the matrix as constructor arguments:

\ifmatlab
\begin{lstlisting}
v = FiniteFieldVariable(prim_poly,3,2);
\end{lstlisting}

This would create a 3x2 matrix of finite field Variables with the given primitive polynomial.

\fi

\ifjava
\begin{lstlisting}
FiniteFieldVariable v = new FiniteFieldVariable(prim_poly);
\end{lstlisting}

This would create a finite field Variable with the given primitive polynomial.

\fi

\para{Addition}

Users can use the following syntax to create an addition factor node with three variables:

\ifmatlab
\begin{lstlisting}
myFactorGraph.addFactor('FiniteFieldAdd',x,y,z);
\end{lstlisting}

using the 'FiniteFieldAdd' built-in factor.
\fi

\ifjava
\begin{lstlisting}
myFactorGraph.addFactor("FiniteFieldAdd",x,y,z);
\end{lstlisting}

The factor function must have the "FiniteFieldAdd" name in order for the SumProduct solver to use the correct custom factor.
\fi

Adding this variable take $\mathcal{O}(1)$ time and solving takes $\mathcal{O}(N\log N)$ time, where N is the size of the finite field domain.

\para{Multiplication}

Similarly, the following syntax can be used to create a factor node with three variables for multiplication:

\ifmatlab
\begin{lstlisting}
myFactorGraph.addFactor('FiniteFieldMult',x,y,z);
\end{lstlisting}
\fi

\ifjava
\begin{lstlisting}
myFactorGraph.addFactor("FiniteFieldMult",x,y,z);
\end{lstlisting}
\fi

Under the hood this will create one of two custom factors, CustomFiniteFieldConstMult or CustomFiniteFieldMult. The former will be created if x or y is a constant and the latter will be created if neither is a constant. This allows Dimple to optimize belief propagation so that it runs in O(N) for multiplication by constants and O(Nlog(N)) in the more general case.

\ifmatlab

\para{NVarFiniteFieldPlus}

Suppose we have the finite field equation $ x1+x2+x3+x4=0 $.

We can not express that using the FiniteFieldAdd factor directly, since it accepts only three arguments. However, we can support larger addition constraints by building a tree of these constraints. We do so using the following function:

\begin{lstlisting}
NumVars = 4;
[graph,vars] = getNVarFiniteFieldPlus(prim_poly,NumVars);
\end{lstlisting}

This function takes a primitive polynomial and the number of variables involved in the constraint and builds up a graph such that  $ x_1+...+x_n = 0 $ .

It returns both the graph and the external variables of the graph. This can be used in one of two ways: setting inputs on the variables and solving the graph directly or using this as a nested sub-graph.

\fi

\para{Projection}

Elements of a finite field with base 2 can be represented as polynomials with binary coefficients. Polynomials with binary coefficients can be represented as strings of bits. For instance, $x^3+x+1$ could be represented in binary as 1011. Furthermore, that number can be represented by the (decimal) integer 11. When using finite fields for decoding, we are often taking bit strings and re-interpreting these as strings of finite field elements. We can use the FiniteFieldProjection built-in factor to relate n bits to a finite field variable with a domain of size $2^n$.

The following code shows how to do that:

\ifmatlab
\begin{lstlisting}
args = cell(n*2,1);
for j = 0:n-1
   args{j*2+1} = j;
   args{j*2+2} = bits(n-j);
end

myFactorGraph.addFactor('FiniteFieldProjection',v,args{:});
\end{lstlisting}
\fi

\ifjava
\begin{lstlisting}
Object [] vars = new Object[n*2];
		
for (int j = 0; j < n; j++)
{
	vars[j*2] = j;
	vars[j*2+1] = new Bit();
}
myFactorGraph.addFactor("FiniteFieldProjection",vars);

\end{lstlisting}
\fi

\subsubsection{Primitive Polynomials}

See Wikipedia for a definition. 

\subsubsection{Algorithmics}


Dimple interprets the domains as integers mapping to bit strings describing the coefficients of polynomials. Internally, the FiniteFieldVariable contains functions to map from this representation to a representation of powers of the primitive polynomial. This operation is known as the discrete log. Similarly Dimple FiniteFieldVariables provide a function to map the powers back to the original representation (i.e., an exponentiation operator).

\begin{itemize}
\item The addition code computes $x+y$ by performing a fast Hadamard transform of the distribution of both $x$ and $y$, pointwise multiplying the transforms, and then performing an inverse fast Hadamard transform.
\item The generic multiplication code computes $xy$ by performing a fast Fourier transform on the distribution of the non-zero elements of the distribution, pointwise multiplying the transforms, performing an inverse fast Fourier transform, and then accounting for the zero elements.
% FIXME
\item The constant multiplication code computes $x$ by converting the distribution of the non-zero values of $x$ to the discrete log domain (which corresponds to reshuffling the array), adding the discrete log of   modulo $N-1$ (cyclically shifting the array), and exponentiating (unshuffling the array back to the original representation).  
\end{itemize}
