\subsection{List of Built-in Factors}
\label{sec:builtInFactors}

The following table lists the current set of built-in Dimple factors.  For each, the name is given, followed by the set of variables that would be connected to the factor, followed by any constructor arguments.  Optional variables and constructor arguments are in brackets.  And an arbitrary length list or array of variables is followed by ellipses.  The allowed set of variable data-types for each variable is given in parentheses (B~=~Bit, D~=~Discrete, R~=~Real, C~=~Complex, RJ~=~RealJoint).

\begin{longtable} {p{3.5cm} p{2.2cm} p{2cm} p{7cm}}
Name & Variables & Constructor & Description \\
\hline
\endhead
%
Abs & out(D,R) \newline in(D,R) & [smoothing] & Deterministic absolute value function, where out = abs(in).  An optional smoothing value may be specified as a constructor argument\footnote{\label{ftn:smoothing}If smoothing is enabled, the factor function becomes $e^{-(\textrm{out} - F(\textrm{in}))^2/\textrm{smoothing}}$ (making it non-deterministic) instead of $\delta(\textrm{out} - F(\textrm{in}))$, where $F$ is the deterministic function associated with this factor.  This is useful for solvers that do not work well with deterministic real-valued factors, such as particle BP, particularly when tempering is used.}.\\
%
ACos & out(D,R) \newline in(D,R) & [smoothing] & Deterministic arc-cosine function, where out = acos(in). An optional smoothing value may be specified as a constructor argument$^{\ref{ftn:smoothing}}$. \\
%
AdditiveNoise & out(R) \newline in(B,D,R) & $\sigma$ & Add Gaussian noise with a known standard deviation, $\sigma$, specified in constructor. \\
%
And & out(B) \newline in...(B) & - & Deterministic logical AND function, where out = AND(in...). \\
%
ASin & out(D,R) \newline in(D,R) & [smoothing] & Deterministic arc-sine function, where out = asin(in). An optional smoothing value may be specified as a constructor argument$^{\ref{ftn:smoothing}}$. \\
%
ATan & out(D,R) \newline in(D,R) & [smoothing] & Deterministic arc-tangent function, where out = atan(in). An optional smoothing value may be specified as a constructor argument$^{\ref{ftn:smoothing}}$. \\
%
Bernoulli & $\rho$(R) \newline x...(B) & - & Bernoulli distribution, $p(x|\rho)$, where $\rho$ is a parameter indicating the probability of one, and x is an array of Bit variables.  There can be any number of x variables, all associated with the same parameter value. The conjugate prior for the parameter, $\rho$, is a Beta distribution\footnote{\label{ftn:conugatePrior}It is not necessary to use the conjugate prior, but in some cases there may be a benefit.}. \\
%
Beta & [$\alpha$](R) \newline [$\beta$](R) \newline value...(R) & [$\alpha$] \newline [$\beta$] & Beta distribution. There can be any number of value variables, all associated with the same parameter values.  Parameters $\alpha$ and $\beta$ can be variables, or if both are constant they can be specified in the constructor. \\
%
Categorical & $\alpha$...(RJ)  \newline x...(D) & - & Categorical distribution, $p(x | \alpha)$, where $\alpha$ is a vector of parameter variables and x is an array of discrete variables.  The number of elements in $\alpha$ must equal the domain size of x.  There can be any number of x variables, all associated with the same parameter values.  \newline
The $\alpha$ parameters are represented as a (not necessarily normalized) probability vector. The conjugate prior for this representation is a Dirichlet distribution\textsuperscript{\ref{ftn:conugatePrior}}.  \newline
In the current implementation, the domain of the x variable must be zero-based contiguous integers, $0...N-1$\footnote{\label{ftn:zeroBaseRestriction}This limitation may be lifted in a future version.}. \\
%
Categorical\newline EnergyParameters & $\alpha$...(R)  \newline x...(D) & N & Categorical distribution, $p(x | \alpha)$, where $\alpha$ is a vector of parameter variables and x is an array of discrete variables.  The number of elements in $\alpha$ and the domain size of x must equal the value of the constructor argument, N.  There can be any number of x variables, all associated with the same parameter values.  \newline
In this alternative version of the Categorical distribution, the $\alpha$ parameters are represented as energy values, that is, $\alpha = -\log(\rho)$, where $\rho$ are unnormalized probabilities. The conjugate prior for this representation is such that each entry of $\alpha$ is independently distributed according to a negative exp-Gamma distribution, all with a common $\beta$ parameter\textsuperscript{\ref{ftn:conugatePrior}}.  \newline
In the current implementation, the domain of the x variable must be zero-based contiguous integers, $0...N-1$\textsuperscript{\ref{ftn:zeroBaseRestriction}}. \\
%
Categorical\newline Unnormalized\newline Parameters & $\alpha$...(R)  \newline x...(D) & N & Categorical distribution, $p(x | \alpha)$, where $\alpha$ is a vector of parameter variables and x is an array of discrete variables.  The number of elements in $\alpha$ and the domain size of x must equal the value of the constructor argument, N.  There can be any number of x variables, all associated with the same parameter values.  \newline
In this alternative version of the Categorical distribution, the $\alpha$ parameters are represented as a vector of unnormalized probability values. The conjugate prior for this representation is such that each entry of $\alpha$ is independently distributed according to a Gamma distribution, all with a common $\beta$ parameter\textsuperscript{\ref{ftn:conugatePrior}}.  \newline
In the current implementation, the domain of the x variable must be zero-based contiguous integers, $0...N-1$\textsuperscript{\ref{ftn:zeroBaseRestriction}}. \\
%
ComplexConjugate & out(C) \newline in(C,R) & [smoothing] & Deterministic complex conjugate function, where out = in$^{*}$. An optional smoothing value may be specified as a constructor argument$^{\ref{ftn:smoothing}}$. \\
%
ComplexDivide & quotient(C) \newline dividend(C,R) \newline divisor(C,R) & [smoothing] & Deterministic complex divide function, where $\mathrm{quotient} = \frac{\mathrm{dividend}}{\mathrm{divisor}}$. An optional smoothing value may be specified as a constructor argument$^{\ref{ftn:smoothing}}$. \\
%
ComplexExp & out(C) \newline in(C,R) & [smoothing] & Deterministic complex exponentiation function, where out = exp(in). An optional smoothing value may be specified as a constructor argument$^{\ref{ftn:smoothing}}$. \\
%
ComplexNegate & out(C) \newline in(C,R) & [smoothing] & Deterministic complex negation function, where out = -in. An optional smoothing value may be specified as a constructor argument$^{\ref{ftn:smoothing}}$. \\
%
ComplexProduct & out(C) \newline in...(C,R) & [smoothing] & Deterministic complex product function, where $\mathrm{out} = \prod \mathrm{in}$. An optional smoothing value may be specified as a constructor argument$^{\ref{ftn:smoothing}}$. \\
%
ComplexSubtract & out(C) \newline posIn(C,R) \newline negIn...(C,R) & [smoothing] & Deterministic complex subtraction function, where $\mathrm{out} = \mathrm{posIn} - \sum \mathrm{negIn}$. An optional smoothing value may be specified as a constructor argument$^{\ref{ftn:smoothing}}$. \\
%
ComplexSum & out(C) \newline in...(C,R) & [smoothing] & Deterministic complex summation function, where $\mathrm{out} = \sum \mathrm{in}$. An optional smoothing value may be specified as a constructor argument$^{\ref{ftn:smoothing}}$. \\
%%
%% Primarily for testing, need not be listed here.
%ConstantPower & out(D,R) \newline base(D,R) & power \newline [smoothing] & Deterministic power function, with a constant power. The power value is specified in the constructor. An optional smoothing value may be specified as a constructor argument$^{\ref{ftn:smoothing}}$. \\
%%
%% Primarily for testing, need not be listed here.
%ConstantProduct & out(D,R) \newline in(D,R) & constant \newline [smoothing] &  Deterministic product function, multiplying the input times a constant value.  The constant value is specified in the constructor. An optional smoothing value may be specified as a constructor argument$^{\ref{ftn:smoothing}}$. \\
%
Cos & out(D,R) \newline in(D,R) & [smoothing] & Deterministic cosine function, where out = cos(in). An optional smoothing value may be specified as a constructor argument$^{\ref{ftn:smoothing}}$. \\
%
Cosh & out(D,R) \newline in(D,R) & [smoothing] & Deterministic hyperbolic-cosine function, where out = cosh(in). An optional smoothing value may be specified as a constructor argument$^{\ref{ftn:smoothing}}$. \\
%
Dirichlet & [$\alpha$](RJ) \newline value...(RJ) & [$\alpha$] & Dirichlet distribution.  There can be any number of value variables, all associated with the same parameter values.  Parameter vector $\alpha$ can be a RealJoint variable or a constant specified in the constructor.  The dimension of $\alpha$ and each of the value variables must be identical. \\
%
DiscreteTransition & y(D) \newline x(D) \newline A...(RJ) & - & 
Parameterized discrete transition factor, $p(y | x, A)$, where x and y are discrete variables, and $A$ is a matrix of transition probabilities. The transition matrix is organized such that columns correspond to the output distribution for each input state. That is, the transition matrix multiplies on the left. Each column of A corresponds to a RealJoint variable. The number of columns in A must equal the domain size of x, and the dimension of each element of A  must equal the domain size of y. \newline
Each element of A corresponds to a (not necessarily normalized) probability vector.  The conjugate prior for this representation is such that each element of A is distributed according to a Dirichlet distribution\textsuperscript{\ref{ftn:conugatePrior}}.  \newline
In the current implementation, the domain of the x variable must be zero-based contiguous integers, $0...N-1$\textsuperscript{\ref{ftn:zeroBaseRestriction}}. \\
%
DiscreteTransition\newline EnergyParameters & y(D) \newline x(D) \newline A...(R) & $N_{y}, N_{x} | \newline N$ & 
Parameterized discrete transition factor, $p(y | x, A)$, where x and y are discrete variables, and $A$ is a matrix of transition probabilities. The transition matrix is organized such that columns correspond to the output distribution for each input state. That is, the transition matrix multiplies on the left. The number of columns in A and the domain size of x must equal the value of the constructor argument, $N_{x}$ and the number of rows in A and the domain size of y must equal the value of the constructor argument $N_{y}$.  If $N_{x}$ and $N_{y}$ are equal, a single constructor argument, $N$, may be used.  \newline
The elements of the matrix A are represented as energy values, that is, $A_{i,j} = -\log(\rho_{i,j})$, where $\rho$ is an unnormalized transition probability matrix.  The conjugate prior for this representation is such that each entry of A is independently distributed according to a negative exp-Gamma distribution, all with a common $\beta$ parameter\textsuperscript{\ref{ftn:conugatePrior}}.  \newline
In the current implementation, the domain of the x variable must be zero-based contiguous integers, $0...N-1$\textsuperscript{\ref{ftn:zeroBaseRestriction}}. \\
%
DiscreteTransition\newline Unnormalized\newline Parameters & y(D) \newline x(D) \newline A...(R) & $N_{y}, N_{x} | \newline N$ & 
Parameterized discrete transition factor, $p(y | x, A)$, where x and y are discrete variables, and $A$ is a matrix of transition probabilities. The transition matrix is organized such that columns correspond to the output distribution for each input state. That is, the transition matrix multiplies on the left. The number of columns in A and the domain size of x must equal the value of the constructor argument, $N_{x}$ and the number of rows in A and the domain size of y must equal the value of the constructor argument $N_{y}$.  If $N_{x}$ and $N_{y}$ are equal, a single constructor argument, $N$, may be used.  \newline
The elements of the matrix A are represented as unnormalized probability values.  The conjugate prior for this representation is such that each entry of A is independently distributed according to a Gamma distribution, all with a common $\beta$ parameter\textsuperscript{\ref{ftn:conugatePrior}}.  \newline
In the current implementation, the domain of the x variable must be zero-based contiguous integers, $0...N-1$\textsuperscript{\ref{ftn:zeroBaseRestriction}}. \\
%
Divide & quotient(D,R) \newline dividend(D,R) \newline divisor(D,R) & [smoothing] & Deterministic divide function, where $\mathrm{quotient} = \frac{\mathrm{dividend}}{\mathrm{divisor}}$. An optional smoothing value may be specified as a constructor argument$^{\ref{ftn:smoothing}}$. \\
%
Equality & value...(B,D,R) & [smoothing] & Deterministic equality constraint.  An optional smoothing value may be specified as a constructor argument$^{\ref{ftn:smoothing}}$. \\
%
Equals & out(B) \newline in...(B,D,R) & - & Deterministic equals function, where out~=~(in(1)~==~in(2)~== ... ). \\
%
Exp & out(D,R) \newline in(D,R) & [smoothing] & Deterministic exponentiation function, where out = exp(in). An optional smoothing value may be specified as a constructor argument$^{\ref{ftn:smoothing}}$. \\
%
Gamma & [$\alpha$](R) \newline [$\beta$](R) \newline value...(R) & [$\alpha$] \newline [$\beta$] & Gamma distribution. There can be any number of value variables, all associated with the same parameter values.  Parameters $\alpha$ and $\beta$ can be variables, or if both are constant they can be specified in the constructor. \\
%
GreaterThan & out(B) \newline in1(B,D,R) \newline in2(B,D,R) & - & Deterministic greater-than function, where out = in1 $>$ in2.  \\
%
InverseGamma & [$\alpha$](R) \newline [$\beta$](R) \newline value...(R) & [$\alpha$] \newline [$\beta$] & Inverse Gamma distribution. There can be any number of value variables, all associated with the same parameter values.  Parameters $\alpha$ and $\beta$ can be variables, or if both are constant they can be specified in the constructor. \\
%
LessThan & out(B) \newline in1(B,D,R) \newline in2(B,D,R) & - & Deterministic greater-than function, where out = in1 $<$ in2.  \\
%
LinearEquation & out(D,R) \newline in(B,D,R) & constants \newline [smoothing] & Deterministic linear equation, multiplying an input vector by a constant vector. The constant vector is specified in the constructor.  The number of \emph{in} variables must equal the length of the constant vector. An optional smoothing value may be specified as a constructor argument\textsuperscript{\ref{ftn:smoothing}}. \\
%
Log & out(D,R) \newline in(D,R) & [smoothing] & Deterministic natural log function, where out = log(in). An optional smoothing value may be specified as a constructor argument$^{\ref{ftn:smoothing}}$. \\
%
LogNormal & [$\mu$](R) \newline [$\tau$](R) \newline value...(R) & [$\mu$] \newline [$\tau$] & Log-normal distribution. There can be any number of value variables, all associated with the same parameter values.  Parameters $\mu$ (mean) and $\tau = \frac{1}{\sigma^{2}}$ (precision) can be variables, or if both are constant then fixed parameters can be specified in the constructor. \\
%
MatrixVectorProduct & y(D,R) \newline M(D,R) \newline x(D,R) & $N_{x}$ \newline $N_{y}$ \newline [smoothing] & Deterministic matrix-vector product function, $y = Mx$, where $x$ and $y$ are vectors and $M$ is a matrix. Constructor arguments, $N_{x}$ and $N_{y}$, specify the input and output vector lengths, respectively. The matrix dimension is $N_{y} \times N_{x}$. An optional smoothing value may be specified as a constructor argument$^{\ref{ftn:smoothing}}$. \\
%%
%% Primarily for testing, need not be listed here.
%MixedNormal & value(R) \newline control(B) & $\mu_{0} \newline \tau_{0} \newline \mu_{1} \newline \tau_{1}$ & Simple mixture of two fixed-parameter Normal distributions. The choice of distribution parameters (0 vs. 1) is a function of the control bit. \\
%
Negate & out(D,R) \newline in(D,R) & [smoothing] & Deterministic negation function, where out = -in. An optional smoothing value may be specified as a constructor argument$^{\ref{ftn:smoothing}}$. \\
%
NegativeExpGamma & [$\alpha$](R) \newline [$\beta$](R) \newline value...(R) & [$\alpha$] \newline [$\beta$] & Negative exp-Gamma distribution, which is a distribution over a variable whose negative exponential is Gamma distributed. That is, this is the negative log of a Gamma distributed variable. There can be any number of value variables, all associated with the same parameter values.  Parameters $\alpha$ and $\beta$ can be variables, or if both are constant they can be specified in the constructor, and correspond to the parameters of the underlying Gamma distribution. \\
%
Normal & [$\mu$](R) \newline [$\tau$](R) \newline value...(R) & [$\mu$] \newline [$\tau$] & Normal distribution. There can be any number of value variables, all associated with the same parameter values.  Parameters $\mu$ (mean) and $\tau = \frac{1}{\sigma^{2}}$ (precision) can be variables, or if both are constant then fixed parameters can be specified in the constructor. \\
%
Not & out(B) \newline in(B) & - & Deterministic logical NOT of function, where out = ~in. \\
%
NotEquals & out(B) \newline in...(B,D,R) & - & Deterministic not-equals function, where out~=~$\sim$(in(1)~==~in(2)~== ... ). \\
%
Or & out(B) \newline in...(B) & - & Deterministic logical OR function, where out = OR(in...). \\
%
Power & out(D,R) \newline base(D,R) \newline power(D,R) & [smoothing] & Deterministic power function, where out~=~base~$^{\mathrm{power}}$. An optional smoothing value may be specified as a constructor argument$^{\ref{ftn:smoothing}}$. \\
%
Product & out(D,R) \newline in...(B,D,R) & [smoothing] & Deterministic product function, where $\mathrm{out} = \prod \mathrm{in}$. An optional smoothing value may be specified as a constructor argument$^{\ref{ftn:smoothing}}$. \\
%
Rayleigh & [$\sigma$](R) \newline value...(R) & [$\sigma$] & Rayleigh distribution. There can be any number of value variables, all associated with the same parameter value.  Parameter $\sigma$ can be a variable, or if constant, can be specified in the constructor. \\
%
Sin & out(D,R) \newline in(D,R) & [smoothing] & Deterministic sine function, where out = sin(in). An optional smoothing value may be specified as a constructor argument$^{\ref{ftn:smoothing}}$. \\
%
Sinh & out(D,R) \newline in(D,R) & [smoothing] & Deterministic hyperbolic-sine function, where out = sinh(in). An optional smoothing value may be specified as a constructor argument$^{\ref{ftn:smoothing}}$. \\
%
Sqrt & out(D,R) \newline in(D,R) & [smoothing] & Deterministic square root function, where out = sqrt(in). An optional smoothing value may be specified as a constructor argument$^{\ref{ftn:smoothing}}$. \\
%
Square & out(D,R) \newline in(D,R) & [smoothing] & Deterministic square function, where out = in$^{2}$. An optional smoothing value may be specified as a constructor argument$^{\ref{ftn:smoothing}}$. \\
%
Subtract & out(D,R) \newline posIn(B,D,R) \newline negIn...(B,D,R) & [smoothing] & Deterministic subtraction function, where $\mathrm{out} = \mathrm{posIn} - \sum \mathrm{negIn}$. An optional smoothing value may be specified as a constructor argument$^{\ref{ftn:smoothing}}$. \\
%
Sum & out(D,R) \newline in...(B,D,R) & [smoothing] & Deterministic summation function, where $\mathrm{out} = \sum \mathrm{in}$. An optional smoothing value may be specified as a constructor argument$^{\ref{ftn:smoothing}}$. \\
%
Tan & out(D,R) \newline in(D,R) & [smoothing] & Deterministic tangent function, where out = tan(in). An optional smoothing value may be specified as a constructor argument$^{\ref{ftn:smoothing}}$. \\
%
Tanh & out(D,R) \newline in(D,R) & [smoothing] & Deterministic hyperbolic-tangent function, where out = tanh(in). An optional smoothing value may be specified as a constructor argument$^{\ref{ftn:smoothing}}$. \\
%
VonMises & [$\mu$](R) \newline [$\tau$](R) \newline value...(R) & [$\mu$] \newline [$\tau$] & Von Mises distribution. There can be any number of value variables, all associated with the same parameter values.  Parameters $\mu$ (mean) and $\tau = \frac{1}{\sigma^{2}}$ (precision) can be variables, or if both are constant then fixed parameters can be specified in the constructor.  The distribution is non-zero for value variables in the range $-\pi$ to $\pi$. \\
%
Xor & out(B) \newline in...(B) & - & Deterministic logical XOR function, where out = XOR(in...). \\
%
\end{longtable}

\subsubsection{Solver-Specific Built-in Factors}
\label{sec:solverSpecificBuiltInFactors}

In addition to the built-in factors listed above, there are a set of solver-specific built-in factors, also referred to as ``custom factors.''  These factors may also be specified by name in the addFactor call (using either a quoted string or as a function handle).  These are summarized in the following table:

\begin{longtable} {l p{3cm} p{7cm}}
Name & Solver & Description \\
\hline
\endhead
%
FiniteFieldFactor & SumProduct\footnote{\label{ftn:fff} These may also be used for discrete variables with the Gaussian or particle BP solvers, which use the sum-product solver for discrete-only portions of the graph.}  & See section~\ref{sec:finiteFields} \\ 
FiniteFieldProjection & SumProduct$^{\ref{ftn:fff}}$ & See section~\ref{sec:finiteFields} \\ 
FiniteFieldAdd & SumProduct$^{\ref{ftn:fff}}$ & See section~\ref{sec:finiteFields} \\ 
FiniteFieldConstMult & SumProduct$^{\ref{ftn:fff}}$ & See section~\ref{sec:finiteFields} \\ 
FiniteFieldMult & SumProduct$^{\ref{ftn:fff}}$ & See section~\ref{sec:finiteFields} \\ 
multiplexerCPD & SumProduct$^{\ref{ftn:fff}}$ & See section~\ref{sec:multiplexerCPD} \\ 
CustomXor & MinSum & Same as the Xor factor described above, but with a significantly faster implementation. \\ 
\end{longtable}


\ifmatlab

\subsubsection{Factor Creation Utility Functions}

Dimple includes some helper functions to create other built-in factors using a similar syntax to the overloaded MATLAB functions described in section~\ref{sec:overloaded}.  As for other overloaded functions, above, Dimple automatically creates the factors as well as the output variable(s).  Such helper functions are defined for the following built-in distributions:

\begin{itemize}
\item Bernoulli
\item Beta
\item Categorical\footnote{This utility can be used to create either a Categorical or CategoricalUnnormalizedParameters factor, depending on whether the first argument is a RealJoint variable or an array of Real variables.}
\item CategoricalEnergyParameters
\item Dirichlet
\item Gamma
\item InverseGamma
\item LogNormal
\item NegativeExpGamma
\item Normal
\item Rayleigh
\item VonMises
\end{itemize}

For each, the arguments are the parameters of the distribution.  For example:

\begin{lstlisting}
W = Gamma(alpha, beta);
X = Normal(mean, precision);
Y = Categorical(alphaVector);
Z = Rayleigh(sigma);
\end{lstlisting}


The parameters can be variables, constants\footnote{If \emph{all} parameter inputs are constants, then instead of creating both variables and factors, the variables are created, and the Input to each variable is set accordingly.}, or some of each\footnote{For cases where a parameter is a vector of real values, the parameter may be a Real variable array, a cell-arrays that may mix Real variables and constants, or an array of constants.}.

By default, calling one of these functions creates a single output variable, and the factor is added to the most-recently created graph. �But, optional arguments allow you to specify the dimensions of the array of output variables, or to specify the factor graph. �These arguments can be in either order after the parameters. �For example:

\begin{lstlisting}
W = Gamma(alpha, beta, altGraph);
X = Normal(mean, precision, [100, 1]);
Y = Categorical(alphaVector, [10, 10, 2], aGraph);
Z = Rayleigh(sigma, myGraph, size(somethingElse));
\end{lstlisting}

\fi


Dimple includes some built-in helper functions to create structured graphs, combining smaller factors to form an efficient implementation of a larger subgraph.  Specifically, the following functions are provided:

\ifmatlab
\begin{itemize}
\item getNBitXorDef(n), where n is a positive integer. Returns a nestable graph and an array of n-Bit connector variables. Efficient tree implementation of the XORDelta function.
\item getVXOR(n), where n is a positive integer. Returns a nestable graph and an array of n-Bits connector variables. Constrains exactly one bit to be 1, and all others to be 0.
\item MultiplexerCPD(domains) - See section~\ref{sec:multiplexerCPD}
\item MultiplexerCPD(domain,numZs) - See section~\ref{sec:multiplexerCPD}
\end{itemize}
\fi

\ifjava
\begin{itemize}
\item MultiplexerCPD(domains) - See section~\ref{sec:multiplexerCPD}
\item MultiplexerCPD(domain,numZs) - See section~\ref{sec:multiplexerCPD}
\end{itemize}
\fi

